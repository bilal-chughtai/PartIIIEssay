\documentclass[a4paper,11pt]{article}
\pdfoutput=1 % if your are submitting a pdflatex (i.e. if you have
             % images in pdf, png or jpg format)
\usepackage{jcappub} % for details on the use of the package, please
                     % see the JCAP-author-manual          
\usepackage{enumitem}% http://ctan.org/pkg/enumitem
\usepackage[T1]{fontenc} % if needed

\renewcommand{\v}[1]{\mathbf{#1}}
\newcommand{\Mp}{M_{pl}}
\newcommand{\half}{\frac{1}{2}}
\newcommand{\bphi}{\bar{\phi}}
\newcommand{\ann}[1]{\hat{a}_{\v{#1}}}
\newcommand{\cre}[1]{\hat{a}^\dagger_{\v{#1}}}
\newcommand{\anns}[2]{\hat{a}_{\v{#1}#2}}
\newcommand{\cres}[2]{\hat{a}^\dagger_{\v{#1}#2}}
\newcommand{\vac}{|0>}
\newcommand{\fint}[1]{\int \frac{d^3 #1}{(2\pi)^3}}


\title{\boldmath The Search for CMB B-mode Polarization from Inflationary Gravitational Waves}


%% %simple case: 2 authors, same institution
%% \author{A. Uthor}
%% \author{and A. Nother Author}
%% \affiliation{Institution,\\Address, Country}

% more complex case: 4 authors, 3 institutions, 2 footnotes
\author{B. Chughtai}

% The "\note" macro will give a warning: "Ignoring empty anchor..."
% you can safely ignore it.

\affiliation{University of Cambridge, Cambridge, UK}


% e-mail addresses: one for each author, in the same order as the authors
\emailAdd{bc464@cam.ac.uk}




\abstract{Abstract...}



\begin{document}
\maketitle
\flushbottom


\section{Introduction}

define scale factor
define proper time
define dot and '
units hbar = c = 1
fourier convention

\section{Inflation}

$\v{t}$

Inflation is a brief, but very important, period of accelerated expansion in the very early universe, first proposed by [Guth 1981]. It was initially motivated by three problems with the previous standard big bang cosmology, namely the flatness problem (why was the ratio of energy density and critical density so close to unity), the monopole problem, and the horizon problem (why are seemingly casually disconnected regions of the CMB at the same temperature to very high accuracy). Since the birth of the idea, it has become the leading paradigm to the early universe, as it conveniently also provides a quantum mechanical mechanism of generating the primordial density perturbations seeding cosmological evolution.

\subsection{Inflation Basics}

A flat, homogeneous and isotropic universe is described by the Friedmann–Lemaître–Robertson–Walker (FLRW) metric, which in our sign convention takes form

\begin{equation}
\label{FLRW}
ds^2 = - dt^2 + a^2(t)d\v{x}^2 = a^2(\tau)(-d\tau^2+d\v{x}^2)
\end{equation}

which obeys the Einstein equation $G_{ab} = \Mp^2 T_{ab}$ \footnote{Note throughout this essay we assume the validity of General Relativity to cosmology. One can study inflation without this assumption, as detailed in for example [] ........}, sourced by a perfect fluid with energy momentum tensor $T_{ab}$, which by homogeneity and isotropy must take form

\begin{equation}
\label{densityandpressure}
T_{00} = - \rho(t) \quad T_{0i} = 0 \quad T_{ij} = P(t)g_{ij}
\end{equation}

where we have identify $\rho(t)$ as the total energy density and $P(t)$ as the total pressure, ie summed over all fluid components. For our purposes, we will consider one component, the inflaton field, to dominate.  Substituting \ref{FLRW} and \ref{densityandpressure} into the Einstein Equation we obtain the Friedmann equations

\begin{equation}
H^2 = (\frac{\dot{a}}{a})^2 = \frac{1}{3\Mp^2}\rho
\tag{F1}
\label{F1}
\end{equation}
\begin{equation}
\frac{\ddot{a}}{a} = -\frac{1}{6\Mp^2}(\rho + 3P)
\tag{F2}
\label{F2}
\end{equation}

The condition for inflation to occur is \textit{accelerated expansion}, ie $\ddot{a} >0$. We define the first hubble slow roll parameter 

\begin{equation}
\label{epsilon}
\epsilon := -\frac{\dot{H}}{H^2} = -\frac{d\ln{H}}{d\ln{a}} = \frac{3}{2}(1+\frac{P}{\rho}
\end{equation}

where the last inequality follows from the Friedmann Equations. We find $\ddot{a} >0$ is equivalent to $\epsilon<1$ and to the condition on the equation of state parameter $\omega=P/\rho < -1/3$. 

In order to solve the horizon problem we require inflation to persist for a long duration of time, so $\epsilon$ to remain small. We parametrise how quickly $\epsilon$ changes in the second hubble slow roll parameter 

\begin{equation}
\eta = -\frac{\dot{\epsilon}}{H\epsilon} = -\frac{d\ln{\epsilon}}{d\ln{a}}
\end{equation}

\subsection{Single Scalar Field Dynamics}

The simplest class of inflation models are those consisting of a single scalar field, slowly rolling down its potential. These postulate the existence of a scalar ``inflation'' field $\phi(t,\v{x})$ with lagrangian density $\mathcal{L} = -\half \partial^\mu \phi \partial_\mu \phi - V(\phi)$ and energy momentum tensor $T_{\mu\nu}= \partial)_\mu \phi \partial_\nu \phi - g_{\mu\nu}\mathcal{L}$. \\

Here we consider the classical background evolution, ie take $\phi(t,\v{x}) = \bphi(t)$. There is of course no reason why the field should not also fluctuate spatially, which we consider in the next section. From \ref{densityandpressure} we see that 


SOME MISTAKE HERE OR IN PREVIOUS DEF
\begin{align}
\rho_\phi &=-T^0_0 = \half \dot{\bphi}^2+V(\bphi)\\
P_\phi&=\frac{1}{3}T^i_i = \half \dot{\bphi}^2-V(\bphi)
\end{align}

Inserting these into the Friedmann equations we get the Klein Gordon Equation

\begin{equation}
\tag{KG}
\ddot{\bphi}+3H\dot{\bphi}=-V_{,\phi}
\end{equation}

We also find that by \ref{epsilon} that 

\begin{equation}
\epsilon = \frac{1}{\Mp^2}\frac{\half\dot{bphi^2}}{H^2} < 1
\end{equation}

\subsection{Slow Roll}

The slow roll approximation postulates the kinetic energy and acceleration of the background field is much smaller than its potential energy, which can be encapsulated in terms of our slow roll parameters as $(\epsilon, \eta << 1)$. In this approximation we get by \ref{F1} and \ref{KG} 

\begin{align}
H^2 &\approx \frac{V}{3\Mp^2} \\
3H\dot{\bphi} &\approx -V_{,\phi}
\end{align}

from which we see 

\begin{equation}
\epsilon \approx \half\Mp^2 (\frac{V'}{V})^2 := \epsilon_V
\end{equation}

where we have defined the first \textit{potential} slow roll parameter. We can anologously define a second potential slow roll parameter via

\begin{equation}
\eta_V = \Mp^2 \frac{V''}{V} \approx 2\epsilon - \half\eta
\end{equation}

From here we may calculate the number of e folds of inflation from some time to until the end of inflation. 

\begin{equation}
N(t) := \ln{\frac{a(t_{end}}{a(t)}} = \int_a^a(t) d(\ln{a}) = \int_t^{t_{end}} HDt = \int_{\bphi(t)}^{\bphi_end} \frac{d\bphi}{\sqrt{2\epsilon_V}\Mp}
\label{efolds}
\end{equation}

using $Hdt=\frac{H}{\dot{\bphi}}d\bphi=\frac{d\bphi}{\sqrt{2\epsilon_V}\Mp}$

\subsection{Quantum Fluctuations to $\phi$}

If the inflaton can vary in time, it can also vary in space. The discussion here follows [Baumann]. We consider pertubations over a background

\begin{equation}
\phi(\v{x},\tau) = \bphi(\tau) + \frac{f(\tau, \v{x})}{a(\tau})
\label{phiexpand}
\end{equation}

We begin with the action for the inflaton, minimally coupled to the metric.

\begin{equation}
S =\int d\tau d^3x \mathcal{L}  =  \int d\tau d^3x \sqrt{-g} (-\half g^{\mu \nu}\partial_\mu \phi\partial_\nu \phi - V(\phi)  
\end{equation}

Plugging in the unperturbed FLRW metric we get 


\begin{equation}
S = \int d\tau d^3x \mathcal{L}  = \int d\tau d^3x \half a^2 [(\phi ' )^2 -(\nabla \phi)^2]-a^4V(\phi)
\label{scalarfieldaction}
\end{equation}

We now plug in \ref{phiexpand} and expand to 2nd order in $f$. The first order piece just gives the Klein Gordon for the background field (in conformal time), as expected. We get

\begin{equation}
S^{(2)} = \half \int d\tau d^3x (f')^2 - (\nabla f)^2 + (\frac{a''}{a}-a^2V'')f^2
\end{equation}

after integrating by parts and making use of \ref{F2} in conformal time. In the slow roll approximation we may drop the potential term since:

NEED TO ADD

and so 

\begin{equation}
S^{(2)} \approx \half \int d\tau d^3x (f')^2 - (\nabla f)^2 + \frac{a''}{a}f^2
\end{equation}

Note since we have dropped the potential entirely, this is just the second order action for a massless scalar field. Integrating by parts and demanding $S^{(2)}=0$ gives the Muhkanov Sasaki equation, which we can write in real or fourier space:

\begin{align}
f''-\nabla^2f-\frac{a''}{a}f &= 0 \\
\Leftrightarrow f''_{\v{k}} + (k^2-\frac{a''}{a})f_{\v{k}} &= 0
\label{MS}
\end{align}

We treat these perturbations $f$ quantum mechanically, and so require the technqiques of QFT on curved spacetimes. We'll outline the key steps in quantising this system.

The conjugate momentum to $f$ is $\pi(\tau, \v{x}) =  \frac{\partial \mathcal{L}}{\partial f'} = f'$ using \ref{scalarfieldaction}. We promote these to operators $\hat{f}(\tau, \v{x})$ and $\hat{\pi}(\tau, \v{x})$ satisfying equal time commutation relations $[\hat{f}(\tau, \v{x}), \hat{\pi}(\tau, \v{x'}] = i\delta(\v{x}-\v{x'}$ or equivalently in fourier space $[\hat{f}_{\v{k}}(\tau), \hat{\pi}_{\v{k'}}(\tau)] = (2\pi)^3i\delta(\v{k}+\v{k'})$.

We mode expand $\hat{f}_{\v{k}}(\tau) = f_k(\tau)\ann{k}+f_k^*(\tau)\cre{k}$, demanding the modefunctions $f_k(\tau)$ and $f_k^*(\tau)$ are two linearly independent solutions of the Muhkanov-Sasaki equation. 

Substiuting into the commutation relations we get 

\begin{equation}
W[f_k,f_k^*]\times[\ann(k), \cre(k)] = (2\pi)^3\delta(\v{k}+\v{k'})
\end{equation}

which after normalising the Wronskian to 1 gives the usual commutator of annhilation and creation operators

\begin{equation}
[\ann{k}, \cre{k}] = (2\pi)^3\delta(\v{k}+\v{k'})
\end{equation}

We can now define the Hilbert space as the usual Fock space formed by unions of n particle states obtained by applying n creation operators to the the vacuum, satisfying 

\begin{equation}
\ann{k}\vac =0 \forall \v{k}
\end{equation}

Note this doesn't completely fix the vacuum, since we have not yet fixed our mode functions. We'll construct the Bunch-Davies vacuum, by imposing the mode functions must be positive frequency \footnote{this is required for our hilbert space to only consist of positive norm states, and is a common requirement in many quantum field theories}, and also match the minkowski mode functions $f_k(\tau) \alpha e^{\pm ik\tau}$ at early times, since at early times $\tau \rightarrow -\infty$, $\ref{MS} \rightarrow  f''_{\v{k}} + (k^2)f_{\v{k}} = 0$. 

We further make the quasi-deSitter approximation, where H is constant and $a=-\frac{1}{H\tau}$, and so \ref{MS} becomes

\begin{equation}
f''_k + (k^2-\frac{2}{\tau^2})f_k = 0
\end{equation}

with general solution

\begin{equation}
f_k(\tau) = \alpha \frac{e^{-ik\tau}}{\sqrt{2k}}{(1-\frac{i}{k\tau}} + \beta \frac{e^{ik\tau}}{\sqrt{2k}}{(1+\frac{i}{k\tau}})
\end{equation}

which matches our initial condition for $\beta = 0$ and $\alpha=1$ giving the Bunch Davies mode function 

\begin{equation}
f_k(\tau) = \frac{e^{-ik\tau}}{\sqrt{2k}}{(1-\frac{i}{k\tau}})
\end{equation}

We now have all we require to calcuate the spatial variance, or two point correlator $\varepsilon(0)$ of $f$ and therefore of $\delta \phi = f/a$. First recall the relevant definitions:

the two point correlation function of a field f is given by at fixed (omitted) time $\tau$ by

\begin{equation} 
\varepsilon_f(\v{x})f(\v{y}) = <f(\v{x})f(\v{y})> = <0|f(\v{x}f(\v{y})|0>
\end{equation}

Assuming statistical homogeneity and isotropy, this depends only on $|\v{x}-\v{y}|$, and is related to the power spectrum:

\begin{equation}
<f_{\v{k}}f_{\v{k'}}> = <0|f_{\v{k}}f_{\v{k'}}|0>=(2\pi)^3\delta(\v{k}+\v{k'})P_f(k)\
\end{equation}

by

\begin{equation}
\varepsilon(r)=\varepsilon(\v{x},\v{x}+\v{r})=\fint{k} P_f(k)e^{i\v{k}\cdot\v{r}}
\end{equation}


The power spectrum of our field is easily computed to be 

\begin{equation}
<f_{\v{k}}f_{\v{k'}}> = <(f_k(\tau)\ann{k}+f_k^*(\tau)\cre{k})(f_k(\tau)\ann{k'}+f_k^*(\tau)\cre{k'}> = |f_k|^2
\end{equation}

and so the dimensionless power spectra of interest are  

\begin{align}
\Delta^2_f(k):=\frac{k^3}{2\pi^2}P_f(k) &= \frac{k^3}{2\pi^2}|f_k|^2\\
\Rightarrow \Delta^2_{\delta\phi}(k) &=\frac{k^3}{2\pi^2a^2}|f_k|^2\\
&=\frac{k^2}{4\pi^2a^2}(1+\frac{1}{k^2\tau^2}\\
&=(\frac{H}{2\pi})^2(1+\frac{k^2}{a^2H^2} \text{using $a=\frac{-1}{H\tau}$}\\
&\rightarrow \frac{H}{2\pi})^2 \text{on superhorizon scales $k<<\mathcal{H}$}
\label{inflatonpower}
\end{align}

We approximate the power spectrum at horizon crossing to be 

\begin{equation}
\Delta^2_{\delta\phi}(k) \approx (\frac{H}{2\pi})^2\rvert_{k=aH}
\end{equation}

We will return to this result later.

\subsection{Metric Pertubations}

dsopfuadsiopfuadiosuf

It turns out that $\mathcal{R}$ is the relevant scalar metric pertubation, because it can be shown to be conserved on superhorizon scales, and so is the relevant quantity to consider to source initial conditions for our universe, as it remains frozen until late times when it renters the horizon. It may be related to the inflaton field rather simply as 

\begin{equation}
\mathcal{R} = -\frac{\mathcal{H}}{\bphi'}\delta\phi
\end{equation}

and so 

\begin{align}
\Delta^2_{\mathcal{R}}(k) &=(\frac{\mathcal{H}}{\bphi'})^2\Delta^2_{\delta\phi}(k)\\
&=\frac{1}{2\epsilon\Mp^2}(\frac{H}{2\pi})^2\rvert_{k=aH}
\end{align}

by \ref{epsilon}


........



 In principle we should consider the full action of the matter sector and the Einstein Hilbert term, though it turns out in spatially flat gauge the metric pertubations are slow roll supressed by a factor of $\epsilon$, and so we may consider only the matter sector. The Einstein-Hilbert term will become important later when we consider tensor perturbations to the metric.\\




\subsection{Gravitational Wave Background}

The remaining piece of the most general metric pertubation we must still consider are the two degrees of freedom in tensor pertubations. Considering just these we have

\begin{equation}
ds^2 = a^2(\tau)(-d\tau^2 + (\delta_{ij}+\gamma_{ij})dx^idx^j)
\end{equation}

where $\gamma_ij$ is symmetric, transverse, and traceless: ie $\gamma_{ij}=\gamma_{ji}$, $\partial_i\gamma_{ij}=0$ and $\gamma_{ii}=0$.

As previously, we need to insert this into the action, which here is the Einstein-Hilbert action, and expand to second order in pertubations. This calculation is rather long, so we omit the details here, but one can find them at gjkdhfghsdfg. 

\begin{equation}
S=\frac{\Mp^2}{2} \int d\tau d^3x \sqrt{-g}R \rightarrow S^{(2)} = \frac{\Mp^2}{8}\int d\tau d^3x \gamma_{ij}'\gamma_{ij}'-\partial_k\gamma_{ij}\partial_k\gamma_{ij}
\end{equation}

By the symmetries of the problem these are the only terms we expect to appear at second order, though one has to go through the calculation to get the correct numerical factors as they turn out to be important.

For the scalar case we were able to deduce the classical equation of motion directly from the second order action, but here it will be easier to work in fourier space from the get go. We therefore expand the graviton in plane waves

\begin{equation}
\gamma_{ij}(\tau, \v{x}) = \fint{k} \sum_{s=+,x} \epsilon_{ij}^s(\v{k})\gamma_s(\tau,\v{k})e^{i\v{k}\cdot\v{x}}
\end{equation}

where $\epsilon_{ij}^s$ are in general complex polarisation tensors satisfying


\begin{align}
\epsilon_{ii}^s(\v{k}) &= k^i \epsilon_{ij}^s(\v{k}) = 0 &\text{transverse and traceless}\\
\epsilon_{ij}^s(\v{k}) &= \epsilon_{ji}^s(\v{k}) &\text{symettric}\\
\epsilon_{ij}^s(\v{k})\epsilon_{jk}^s(\v{k}) &= 0&\text{null}\\
\epsilon_{ij}^s(\v{k})\epsilon_{ij}^s(\v{k})^* &= 2\delta_{ss'} &\text{normalisation}\\
\epsilon_{ij}^s(\v{k})^* &= \epsilon_{ij}^s(\v{-k})&\text{$\gamma_{ij}$ real}
\end{align}

we may insert this expansion into $S^{(2)}$ and obtain MAYBE INCLUDE THE STEPS

\begin{equation}
S^{(2)} = \frac{\Mp^2}{2} \half \fint{k} d\tau a^2 \sum_{s=+,x} \gamma_s'(\tau,\v{k})\gamma_s '(\tau,\v{-k})+k^2 \gamma_s(\tau,\v{k})\gamma_s (\tau,\v{-k})
\label{gravwaveaction}
\end{equation}

We find this is, up to a constant, really just two copies of a special case of the action \ref{scalarfieldaction}. To see this, consider \ref{scalarfieldaction} in the special case where $\bphi = V(\bphi) = 0$, i.e. $\phi = \delta \phi$ so we have 


\begin{equation}
S = \half \int d\tau d^3x a^2 [(\phi ' )^2 -(\nabla \phi)^2]
\end{equation}

which, in fourier space $\phi = \fint{k} \phi(\v{k})e^{i\v{k}\cdot\v{x}}$ is just

\begin{equation}
S = \half \int d\tau d^3x a^2 [\phi'(\v{k})\phi'(\v{-k}) + k^2 \phi(\v{k})\phi(\v{-k})]
\end{equation}

comparing with \ref{gravwaveaction} we see we may quantise the two independent fields $\tilde{\gamma_s} = \frac{\Mp a}{\sqrt{2}} \gamma_s$ exactly as we did $ f = a \delta \phi$. The normalisation is required to give canonical factor of $\half$ in the action, which plays a role when we fix the Wronskian to 1. Going through the same procedure as before, we get:

operators $\hat{\gamma}_s(\v{k}) = \frac{\sqrt{2}\Mp}{a}(f_k\anns{k}{s}+f_k^*\cres{k}{s})$

with commutation relations $[\anns{k}{s},\cres{k'}{s}] = (2\pi)^3\delta(\v{k}-\v{k'})\delta_{ss'}$

where the mode functions $f_k$ are the same Bunch Davies mode functions as last time.


The final result of this section is to calculate the tensor power spectrum. 
We have

\begin{align}
<\gamma_{ij}(\v{k})\gamma_{ij}(\v{k'})> & = \sum_{ss'} \epsilon^s_{ij}(\v{k})\epsilon^s_{ij}(\v{k'})<\gamma_{s}(\v{k})\gamma_{s}(\v{k'})>\\
&= (\frac{\sqrt{2}\Mp}{a})^2 \sum_{ss'} \epsilon^s_{ij}(\v{k})\epsilon^s_{ij}(\v{k'})(2\pi)^3\delta(\v{k}+\v{k'})|f_k|^2\\
&= (\frac{\sqrt{2}\Mp}{a})^2 \sum_{ss'} 2\delta_{ss'}(2\pi)^3\delta(\v{k}+\v{k'})|f_k|^2\\
&= (2\pi)^3\delta(\v{k}+\v{k'})\frac{8}{\Mp^2a^2}|f_k|^2
\end{align}

From which we can read off 

\begin{equation}
P_t(k)=\frac{8}{\Mp^2}P_{\delta\phi}(k) \Rightarrow \Delta^2_t(k)=\frac{8}{\Mp^2}\Delta^2_{\delta\phi}(k) = \frac{2}{\pi^2}(\frac{H}{\Mp})^2
\end{equation}


\subsection{Properties of the scalar and tensor power spectra}

In summary, we have calculated the scalar and tensor power spectra:

\begin{equation}
\Delta^2_{\mathcal{R}}(k) = \frac{1}{2\epsilon\Mp^2}(\frac{H}{2\pi})^2\rvert_{k=aH} \qquad
\Delta^2_t(k)\frac{2}{\pi^2}(\frac{H}{\Mp})^2\rvert_{k=aH}
\end{equation}

These freeze out after horizon exit, and thus provide the initial conditions of the universe. Given a good understanding of the physics of proceeding cosmological evolution we have several direct probes of these power spectra. The remainder of this essay will go into more detail about how we do this this: we will explicitly calculate the effect these scalar and tensor perturbations have on several CMB observables, thus providing a way to detect these primordial fluctuations.

To provide some constraining power, we convert the information contained into these power spectra into a more useful form. Recall H is approximately constant during inflation, but not perfectly so, since $H^2 \sim V(\phi)$, and $\phi$ is slowly rolling down its potential. Similarly $\epsilon$ also varies slightly. We can parametrise this deviation from scale invariance to first order by the scalar and tensor spectral indexes $n_i$, and also introduce the respective amplitudes $A_i$ via:

\begin{equation}
\Delta^2_{\mathcal{R}}(k) \approx A_s(\frac{k}{k_*})^{n_s-1} \qquad
\Delta^2_t(k) \approx A_t(\frac{k}{k_*})^{n_t}
\end{equation}

where $k_*$ is some pivot scale, often taken to be $0.05Mpc^{-1}$.

We can compute the spectral indexes in terms of the slow roll parameters. 

\begin{align}
n_s-1 &= \frac{d\ln{\Delta^2_{\mathcal{R}}}}{d\ln{k}} = \frac{d}{d\ln{k}}(2\ln{H}-ln{\epsilon} \approx \frac{1}{H}\frac{d}{dt}(2\ln{H}-\ln{\epsilon}) = \frac{2\dot{H}}{H^2}-\frac{\dot{\epsilon}}{H\epsilon} = -2\epsilon-\eta = -6\epsilon_V+2\eta_V\\
n_t &= \frac{d\ln{\Delta^2_t}}{d\ln{k}} = \frac{d}{d\ln{k}}2\ln{H}=-2\epsilon=-2\epsilon_V
\end{align}
noting that that $d\ln{k} = d\ln{aH} \approx d\ln{a} = Hdt$ since we evaluate the power spectra 
at horizon exit, and H is slowly varying (so $d\ln{H}$ provides a next to leading order correction.


During inflation $\epsilon>0$ since energy density is monotonically decreasing, and so $n_t<0$ and tensor pertubations are said to have a red spectrum.

We can also define the scalar to tensor ratio 

\begin{equation}
r=\frac{A_t}{A_s} \approx 16\epsilon
\end{equation}

leading to the consistency condition $r=-8n_t$, serving as an easy test for single field slow roll inflation. Other inflation models predict different consistency conditions.


\subsection{What can we learn?}
We can learn several interesting things about the physics of inflation itself from these. We'll go into some detail about several. It shouldn't be suprising that this information is contained within the parameters we previously defined. Writing $n_s$ in terms of the potential slow roll parameters indicates $n_s$ contains information about the shape of the inflationary potential.\\ 

Recent CMB experiments have managed to measure $n_s$ to good precision to be $n_s=0.968\pm0.006$, differing from a scale invariant $n_s=1$ by over $5\sigma$. [REF] SFSR inflation models can be written down with either a red ($n_s<1$) or blue ($n_s>1$ spectrum, and this measurement therefore rules out a large slew of such models. \\

At present, we have gained no statistically significant evidence for tensor modes, i.e. have no measured value of $n_t$ or $r$, though the lack of detection constrains $r<0.1$ [BICEP 2 KECK + PLANCK]. Upcoming experiments [CMbS4] seek to improve this, though as we will show this constraint is already useful. 

Given the large number of inflation models, even within the SFSR regime, that exist in the literature, it is useful to work model independently. We'll describe some ifnromation we can gather:

\subsubsection{The energy scale of inflation}

In the slow roll approximation, we may write 

\begin{equation}
\Delta^2_{\mathcal{R}}\approx \frac{V}{24\pi^2\epsilon\Mp^4}
\end{equation}

Current measurements [QBM -> PLANCK] give $\Delta^2_{\mathcal{R}} \approx A_s \approx 2.2\times10^{-9}$. Making use of $r\approx 16\epsilon$, we learn

\begin{equation}
V=24\pi^2A_s\frac{r}{16}\Mp^4 \Rightarrow V^{1/4} \approx 3.12\times10^{16}\text{GeV}r^{1/4} \leq 1.75\times10^{16}\text{GeV}
\end{equation}

This can be further improved with more accurate bounds on r.

\subsubsection{Field excursion}
Recall \ref{efolds} giving 

\begin{equation}
N(t) =  \int_{\bphi(t)}^{\bphi_end} \frac{d\bphi}{\sqrt{2\epsilon_V}\Mp}
\Rightarrow dN=\frac{d\bphi}{\sqrt{2\epsilon_V}\Mp}
\Rightarrow \frac{\Delta \phi}{\Mp} = \int_0^N \sqrt{2\epsilon} dN = \int_0^N \sqrt{r/8} dN 
\end{equation}



























\section{CMB}


\section{Some examples and best-practices}
\label{sec:intro}

Here follow some examples of common features that you may wanto to use
or build upon.

For internal references use label-refs: see section~\ref{sec:intro}.
Bibliographic citations can be done with cite: refs.~\cite{a,b,c}.
When possible, align equations on the equal sign. The package
\texttt{amsmath} is already loaded. See \eqref{eq:x}.
\begin{equation}
\label{eq:x}
\begin{split}
x &= 1 \,,
\qquad
y = 2 \,,
\\
z &= 3 \,.
\end{split}
\end{equation}
Also, watch out for the punctuation at the end of the equations.


If you want some equations without the tag (number), please use the available
starred-environments. For example:
\begin{equation*}
x = 1
\end{equation*}

The amsmath package has many features. For example, you can use use
\texttt{subequations} environment:
\begin{subequations}\label{eq:y}
\begin{align}
\label{eq:y:1}
a & = 1
\\
\label{eq:y:2}
b & = 2
\end{align}
and it will continue to operate across the text also.
\begin{equation}
\label{eq:y:3}
c = 3
\end{equation}
\end{subequations}
The references will work as you'd expect: \eqref{eq:y:1},
\eqref{eq:y:2} and \eqref{eq:y:3} are all part of \eqref{eq:y}.

A similar solution is available for figures via the \texttt{subfigure}
package (not loaded by default and not shown here).
All figures and tables should be referenced in the text and should be
placed at the top of the page where they are first cited or in
subsequent pages. Positioning them in the source file
after the paragraph where you first reference them usually yield good
results. See figure~\ref{fig:i} and table~\ref{tab:i}.

\begin{figure}[tbp]
\centering 
\includegraphics[width=.45\textwidth,trim=0 380 0 200,clip]{example-image}
\hfill
\includegraphics[width=.45\textwidth,angle=180]{example-image}
\caption{\label{fig:i} Always give a caption.}
\end{figure}

\begin{table}[tbp]
\centering
\begin{tabular}{|lr|c|}
\hline
x&y&x and y\\
\hline
a & b & a and b\\
1 & 2 & 1 and 2\\
$\alpha$ & $\beta$ & $\alpha$ and $\beta$\\
\hline
\end{tabular}
\caption{\label{tab:i} We prefer to have borders around the tables.}
\end{table}

We discourage the use of inline figures (wrapfigure), as they may be
difficult to position if the page layout changes.

We suggest not to abbreviate: ``section'', ``appendix'', ``figure''
and ``table'', but ``eq.'' and ``ref.'' are welcome. Also, please do
not use \texttt{\textbackslash emph} or \texttt{\textbackslash it} for
latin abbreviaitons: i.e., et al., e.g., vs., etc.



\section{Sections}
\subsection{And subsequent}
\subsubsection{Sub-sections}
\paragraph{Up to paragraphs.} We find that having more levels usually
reduces the clarity of the article. Also, we strongly discourage the
use of non-numbered sections (e.g.~\texttt{\textbackslash
  subsubsection*}).  Please also see the use of
``\texttt{\textbackslash texorpdfstring\{\}\{\}}'' to avoid warnings
from the hyperref package when you have math in the section titles



\appendix
\section{Essay Description}
Our most promising theory for the early universe involves a phase of cosmic inflation, which not only rapidly expands and flattens the universe, but also generates the primordial density perturbations from quantum fluctuations in the inflaton field. While we have good evidence for inflation, e.g. from the Gaussianity, adiabaticity and near-scale invariance of the scalar density perturbations, one prediction of inflation has not yet been found: many inflationary models produce a stochastic background of primordial gravitational waves. A detection of this background would not only provide a definitive confirmation of inflation, but could also give new insights into the microphysics of inflation and, more broadly, physics at the highest energies.\\

The best current way of finding this gravitational wave background is to search for a characteristic pattern in the polarization of the Cosmic Microwave Background (CMB), the B-mode polarization. This essay should explain the physics underlying the search for this B-mode polarization pattern, which is currently a major area of research in cosmology.
The essay should first review the calculation of the gravitational wave background produced
by standard single-field slow-roll inflation, a standard result described in past Part III lecture notes as well as a comprehensive review of the field (Kamionkowski \& Kovetz 2016, henceforth KK16). The essay should also explain why the strength of the gravitational wave background(together with the scalar spectral index) can provide powerful constraints on the properties of inflation, such as the potential shape, energy scale, and field excursion (CMB-S4 2016, KK16).\\

Drawing on KK16, CMB-S4 2016, past lecture notes and other resources, the essay should provide a (brief) review of the basics of CMB polarization, describe what the CMB B-mode polarization is, and explain why it is a powerful probe of inflationary gravitational waves.\\

The remaining parts of the essay can, to some extent, be tailored to the student’s interests. One option is to explain in detail the major observational challenges in B-mode searches for inflationary gravitational waves, discussing the problems of foregrounds (Bicep/Keck/Planck 2015) and gravitational lensing as well as mitigation methods such as multifrequency cleaning and delensing (Smith et al. 2012). Another option is to focus more on the theoretical background, describing in detail different classes of inflationary models and what these generically predict for B-mode polarization (CMB-S4 2016 and references therein). Students may also discuss a combination of both observational and theoretical aspects.\\

\textbf{Relevant Courses}\\

\textit{Essential}: Cosmology\\

\textit{Useful}: Advanced Cosmology, Quantum Field Theory, General Relativity\\

\textbf{References}
\begin{enumerate}[label={[\arabic*]}]
\item {Kamionkowski, M. \& Kovetz, E. D. 2016, Annual Review of Astronomy and Astrophysics,
54, 227}
\item {CMB-S4 Science Book 2016, arXiv:1610.02743 (mainly chapter 2)}
\item {BICEP/Keck/Planck 2015, arXiv:1502.00612, Phys. Rev. Lett. 141 101301}
\item {Smith, K. M. et al. 2012, arXiv:1010.0048, JCAP, 06 014}
\item {Baumann, D., lecture notes: http://www.damtp.cam.ac.uk/user/db275/Cosmology/Lectures.pdf}
\end{enumerate}


\acknowledgments

This is the most common positions for acknowledgments. A macro is
available to maintain the same layout and spelling of the heading.

\paragraph{Note added.} This is also a good position for notes added
after the paper has been written.





% The bibliography will probably be heavily edited during typesetting.
% We'll parse it and, using the arxiv number or the journal data, will
% query inspire, trying to verify the data (this will probalby spot
% eventual typos) and retrive the document DOI and eventual errata.
% We however suggest to always provide author, title and journal data:
% in short all the informations that clearly identify a document.

\begin{thebibliography}{99}

\bibitem{a}
Author, \emph{Title}, \emph{J. Abbrev.} {\bf vol} (year) pg.

\bibitem{b}
Author, \emph{Title},
arxiv:1234.5678.

\bibitem{c}
Author, \emph{Title},
Publisher (year).


% Please avoid comments such as "For a review'', "For some examples",
% "and references therein" or move them in the text. In general,
% please leave only references in the bibliography and move all
% accessory text in footnotes.

% Also, please have only one work for each \bibitem.


\end{thebibliography}
\end{document}
