\documentclass[a4paper,11pt]{article}
\pdfoutput=1 % if your are submitting a pdflatex (i.e. if you have
             % images in pdf, png or jpg format)
\usepackage{jcappub} % for details on the use of the package, please
                     % see the JCAP-author-manual          
\usepackage{enumitem}% http://ctan.org/pkg/enumitem
\usepackage[T1]{fontenc} % if needed

\renewcommand{\v}[1]{\mathbf{#1}}
\newcommand{\Mp}{M_{pl}}
\newcommand{\half}{\frac{1}{2}}
\newcommand{\bphi}{\bar{\phi}}
\newcommand{\ann}[1]{\hat{a}_{\v{#1}}}
\newcommand{\cre}[1]{\hat{a}^\dagger_{\v{#1}}}
\newcommand{\anns}[2]{\hat{a}_{\v{#1}#2}}
\newcommand{\cres}[2]{\hat{a}^\dagger_{\v{#1}#2}}
\newcommand{\vac}{|0\rangle}
\newcommand{\fint}[1]{\int \frac{d^3 #1}{(2\pi)^3}}
\newcommand{\unit}[1]{\hat{\v{#1}}}
\newcommand{\sr}{\text{\normalfont\dh}}
\renewcommand{\sl}{\bar{\text{\normalfont\dh}}}
\newcommand{\ltwo}{[\frac{(l+2)!}{(l-2)!}]}
\newcommand{\ltwof}{[\frac{(l-2)!}{(l+2)!}]}

\title{\boldmath The Search for CMB B-mode Polarization from Inflationary Gravitational Waves}


%% %simple case: 2 authors, same institution
%% \author{A. Uthor}
%% \author{and A. Nother Author}
%% \affiliation{Institution,\\Address, Country}

% more complex case: 4 authors, 3 institutions, 2 footnotes
\author{B. Chughtai}

% The "\note" macro will give a warning: "Ignoring empty anchor..."
% you can safely ignore it.

\affiliation{University of Cambridge, Cambridge, UK}


% e-mail addresses: one for each author, in the same order as the authors
\emailAdd{bc464@cam.ac.uk}




\abstract{Abstract...}



\begin{document}
\maketitle
\flushbottom


\section{Introduction}

define scale factor
define proper time
define dot and '
units hbar = c = 1
fourier convention

\section{Inflation}

$\v{t}$

Inflation is a brief, but very important, period of accelerated expansion in the very early universe, first proposed by [Guth 1981]. It was initially motivated by three problems with the previous standard big bang cosmology, namely the flatness problem (why was the ratio of energy density and critical density so close to unity), the monopole problem, and the horizon problem (why are seemingly casually disconnected regions of the CMB at the same temperature to very high accuracy). Since the birth of the idea, it has become the leading paradigm to the early universe, in part because it provides a quantum mechanical mechanism of generating the primordial density perturbations seeding cosmological evolution. In this chapter we quickly review some important features of inflation.

\subsection{Inflation Basics}

A flat, homogeneous and isotropic universe is described by the Friedmann–Lemaître–Robertson–Walker (FLRW) metric, which in our sign convention takes form

\begin{equation}
\label{FLRW}
ds^2 = - dt^2 + a^2(t)d\v{x}^2 = a^2(\tau)(-d\tau^2+d\v{x}^2)
\end{equation}

and, assuming General Relativity, obeys the Einstein equation $G_{ab} = \frac{1}{\Mp^2} T_{ab}$, sourced by a perfect fluid with energy momentum tensor $T_{ab}$, which by homogeneity and isotropy must take form

\begin{equation}
\label{densityandpressure}
T^0_0 = - \rho(t) \quad T^0_i = 0 \quad T^i_j = P(t)\delta^i_j
\end{equation}

where we  identify $\rho(t)$ as the total energy density and $P(t)$ as the total pressure, ie summed over all fluid components. For our purposes, we will consider one component, the inflaton field, to dominate.  Substituting \ref{FLRW} and \ref{densityandpressure} into the Einstein Equation we obtain the Friedmann equations

\begin{equation}
H^2 = (\frac{\dot{a}}{a})^2 = \frac{1}{3\Mp^2}\rho
\tag{F1}
\label{F1}
\end{equation}
\begin{equation}
\frac{\ddot{a}}{a} = -\frac{1}{6\Mp^2}(\rho + 3P)
\tag{F2}
\label{F2}
\end{equation}

The condition for inflation to occur is \textit{accelerated expansion}, ie $\ddot{a} >0$. Recall the definition of the first hubble slow roll parameter 

\begin{equation}
\label{epsilon}
\epsilon := -\frac{\dot{H}}{H^2} = -\frac{d\ln{H}}{d\ln{a}} = \frac{3}{2}(1+\frac{P}{\rho})
\end{equation}

where the last inequality follows from the Friedmann Equations. We find $\ddot{a} >0$ is equivalent to $\epsilon<1$ and to the condition on the equation of state parameter $\omega=P/\rho < -1/3$.\\ 

In order to solve the horizon problem we require inflation to persist for a relatively long duration of time (~60 e-folds), so $\epsilon$ to must remain small. We parametrise how quickly $\epsilon$ changes in the second hubble slow roll parameter 

\begin{equation}
\eta = -\frac{\dot{\epsilon}}{H\epsilon} = -\frac{d\ln{\epsilon}}{d\ln{a}}
\end{equation}

\subsection{Single Scalar Field Dynamics}

The simplest class of inflation models are those consisting of a single scalar field, slowly rolling down its potential. These postulate the existence of a scalar ``inflation'' field $\phi(t,\v{x})$ with lagrangian density $\mathcal{L} = -\half \partial^\mu \phi \partial_\mu \phi - V(\phi)$ and energy momentum tensor $T_{\mu\nu}= \partial_\mu \phi \partial_\nu \phi + g_{\mu\nu}\mathcal{L}$. \\

Here we consider the classical background evolution, ie take $\phi(t,\v{x}) = \bphi(t)$. There is of course no reason why the field should not also fluctuate spatially, which we consider in the next section. From \ref{densityandpressure} we see that 


SOME MISTAKE HERE OR IN PREVIOUS DEF
\begin{align}
\rho_\phi &=-T^0_0 = \half \dot{\bphi}^2+V(\bphi)\\
P_\phi&=\frac{1}{3}T^i_i = \half \dot{\bphi}^2-V(\bphi)
\end{align}

Inserting these into the Friedmann equations we get the Klein Gordon Equation

\begin{equation}
\tag{KG}
\label{KG}
\ddot{\bphi}+3H\dot{\bphi}=-V_{,\phi}
\end{equation}

We also find that by \ref{epsilon} that 

\begin{equation}
\epsilon = \frac{1}{\Mp^2}\frac{\half\dot{\bphi^2}}{H^2} < 1
\end{equation}

\subsection{Slow Roll}

The slow roll approximation postulates the kinetic energy and acceleration of the background field is much smaller than its potential energy, which can be encapsulated in terms of our slow roll parameters as $(\epsilon, \eta << 1)$. In this approximation we get by \ref{F1} and \ref{KG} 

\begin{align}
H^2 &\approx \frac{V}{3\Mp^2} \\
3H\dot{\bphi} &\approx -V_{,\phi}
\end{align}

from which we see 

\begin{equation}
\epsilon \approx \half\Mp^2 (\frac{V'}{V})^2 := \epsilon_V
\end{equation}

where we have defined the first \textit{potential} slow roll parameter. We can anologously define a second potential slow roll parameter via

\begin{equation}
\eta_V = \Mp^2 \frac{V''}{V} \approx 2\epsilon - \half\eta
\end{equation}

From here we may calculate the number of e folds of inflation from some time $t$ until the end of inflation. 

\begin{equation}
N(t) := \ln{\frac{a(t_{end})}{a(t)}} = \int_a^{a(t)} d(\ln{a}) = \int_t^{t_{end}} Hdt = \int_{\bphi(t)}^{\bphi_{end}} \frac{d\bphi}{\sqrt{2\epsilon_V}\Mp}
\label{efolds}
\end{equation}

using $Hdt=\frac{H}{\dot{\bphi}}d\bphi=\frac{d\bphi}{\sqrt{2\epsilon_V}\Mp}$

\subsection{Quantum Fluctuations to $\phi$}

If the inflaton can vary in time, it can also vary in space. The discussion here follows [Baumann]. We consider pertubations over a background

\begin{equation}
\phi(\v{x},\tau) = \bphi(\tau) + \frac{f(\tau, \v{x})}{a(\tau)}
\label{phiexpand}
\end{equation}

We begin with the action for the inflaton, minimally coupled to the metric.

\begin{equation}
S =\int d\tau d^3x \mathcal{L}  =  \int d\tau d^3x \sqrt{-g} (-\half g^{\mu \nu}\partial_\mu \phi\partial_\nu \phi - V(\phi)  
\end{equation}

Plugging in the unperturbed FLRW metric we get 


\begin{equation}
S = \int d\tau d^3x \mathcal{L}  = \int d\tau d^3x \half a^2 [(\phi ' )^2 -(\nabla \phi)^2]-a^4V(\phi)
\label{scalarfieldaction}
\end{equation}

We now plug in \ref{phiexpand} and expand to 2nd order in $f$. The first order piece just gives the Klein Gordon for the background field (in conformal time), as expected. The second order piece gives

\begin{equation}
S^{(2)} = \half \int d\tau d^3x (f')^2 - (\nabla f)^2 + (\frac{a''}{a}-a^2V'')f^2
\end{equation}

after integrating by parts and making use of \ref{F2} in conformal time. In the slow roll approximation we may drop the potential term since it is slow roll suppressed compared to the other terms.

and so 

\begin{equation}
S^{(2)} \approx \half \int d\tau d^3x (f')^2 - (\nabla f)^2 + \frac{a''}{a}f^2
\end{equation}

Note since we have dropped the potential entirely, this is just the second order action for a massless scalar field. Integrating by parts and demanding $S^{(2)}=0$ gives the Muhkanov Sasaki equation, which we can write in real or fourier space:

\begin{align}
f''-\nabla^2f-\frac{a''}{a}f &= 0 \\
\Leftrightarrow f''_{\v{k}} + (k^2-\frac{a''}{a})f_{\v{k}} &= 0
\label{MS}
\end{align}

We treat these perturbations $f$ quantum mechanically, and so require the techniques of QFT on curved spacetimes. We'll outline the key steps in quantising this system.\\

The conjugate momentum to $f$ is $\pi(\tau, \v{x}) =  \frac{\partial \mathcal{L}}{\partial f'} = f'$ using \ref{scalarfieldaction}. We promote these to operators $\hat{f}(\tau, \v{x})$ and $\hat{\pi}(\tau, \v{x})$ satisfying equal time commutation relations which read in real and fourier space:

\begin{align}
[\hat{f}(\tau, \v{x}), \hat{\pi}(\tau, \v{x'}] &= i\delta(\v{x}-\v{x'}) 
[\hat{f}_{\v{k}}(\tau), \hat{\pi}_{\v{k'}}(\tau)] &= (2\pi)^3i\delta(\v{k}+\v{k'})
\end{align}

We mode expand $\hat{f}_{\v{k}}(\tau) = f_k(\tau)\ann{k}+f_k^*(\tau)\cre{k}$, demanding the modefunctions $f_k(\tau)$ and $f_k^*(\tau)$ are two linearly independent solutions of the Muhkanov-Sasaki equation. Substituting into the commutation relations we get 

\begin{equation}
W[f_k,f_k^*]\times[\ann{k}, \cre{k}] = (2\pi)^3\delta(\v{k}+\v{k'})
\end{equation}

which after normalising the Wronskian to 1 gives the usual commutator of annhilation and creation operators

\begin{equation}
[\ann{k}, \cre{k}] = (2\pi)^3\delta(\v{k}+\v{k'})
\end{equation}

We can now define the Hilbert space as the usual Fock space formed by unions of n particle states obtained by applying n creation operators to the the vacuum, satisfying 

\begin{equation}
\ann{k}\vac =0 \qquad \forall \v{k}
\end{equation}

Note this doesn't completely fix the vacuum, since we have not yet fixed our mode functions. We construct the Bunch-Davies vacuum, by imposing the mode functions must be positive frequency \footnote{this is required for our hilbert space to only consist of positive norm states, and is a common requirement in many quantum field theories}, and also match the minkowski mode functions $f_k(\tau) \propto e^{\pm ik\tau}$ at early times, since at early times $\tau \rightarrow -\infty$

\begin{equation}
\ref{MS} \rightarrow  f''_{\v{k}} + (k^2)f_{\v{k}} = 0
\end{equation}

We further make the quasi-deSitter approximation, where H is constant and $a=-\frac{1}{H\tau}$, and so \ref{MS} becomes

\begin{equation}
f''_k + (k^2-\frac{2}{\tau^2})f_k = 0
\end{equation}

with general solution

\begin{equation}
f_k(\tau) = \alpha \frac{e^{-ik\tau}}{\sqrt{2k}}{(1-\frac{i}{k\tau}} + \beta \frac{e^{ik\tau}}{\sqrt{2k}}{(1+\frac{i}{k\tau}})
\end{equation}

which matches our initial condition for $\beta = 0$ and $\alpha=1$, giving the Bunch Davies mode function 

\begin{equation}
f_k(\tau) = \frac{e^{-ik\tau}}{\sqrt{2k}}{(1-\frac{i}{k\tau}})
\end{equation}

The key result of this section is the power spectrum of fluctuations, given by 

%We now have all we require to calcuate the spatial variance, or two point correlator $\varepsilon(0)$ of $f$ and therefore of $\delta \phi = f/a$. First recall the relevant definitions:
%
%the two point correlation function of a field f is given by at fixed (omitted) time $\tau$ by
%
%\begin{equation} 
%\varepsilon_f(\v{x})f(\v{y}) = <f(\v{x})f(\v{y})> = <0|f(\v{x}f(\v{y})|0>
%\end{equation}
%
%Assuming statistical homogeneity and isotropy, this depends only on $|\v{x}-\v{y}|$, and is related to the power spectrum:

\begin{equation}
\langle f_{\v{k}}f_{\v{k'}} \rangle = \langle 0|f_{\v{k}}f_{\v{k'}}|0\rangle =(2\pi)^3\delta(\v{k}+\v{k'})P_f(k)\
\end{equation}

%by
%
%\begin{equation}
%\varepsilon(r)=\varepsilon(\v{x},\v{x}+\v{r})=\fint{k} P_f(k)e^{i\v{k}\cdot\v{r}}
%\end{equation}


The power spectrum of our field is easily computed to be 

\begin{equation}
\langle f_{\v{k}}f_{\v{k'}}\rangle = \langle (f_k(\tau)\ann{k}+f_k^*(\tau)\cre{k})(f_k(\tau)\ann{k'}+f_k^*(\tau)\cre{k'}\rangle = |f_k|^2
\end{equation}

and so the dimensionless power spectra of interest are  

\begin{align}
\Delta^2_f(k):=\frac{k^3}{2\pi^2}P_f(k) &= \frac{k^3}{2\pi^2}|f_k|^2\\
\Rightarrow \Delta^2_{\delta\phi}(k) &=\frac{k^3}{2\pi^2a^2}|f_k|^2\\
&=\frac{k^2}{4\pi^2a^2}(1+\frac{1}{k^2\tau^2})\\
&=(\frac{H}{2\pi})^2(1+\frac{k^2}{a^2H^2} \text{using $a=\frac{-1}{H\tau}$}\\
&\rightarrow \frac{H}{2\pi})^2 \text{on superhorizon scales $k<<\mathcal{H}$}
\label{inflatonpower}
\end{align}

We approximate the power spectrum at horizon crossing to be 

\begin{equation}
\Delta^2_{\delta\phi}(k) \approx (\frac{H}{2\pi})^2\rvert_{k=aH}
\end{equation}

We will return to this result later.

\subsection{Metric Pertubations}

dsopfuadsiopfuadiosuf

It turns out that $\mathcal{R}$ is the relevant scalar metric pertubation, because it can be shown to be conserved on superhorizon scales, and so is the relevant quantity to consider to source initial conditions for our universe, as it remains frozen until late times when it renters the horizon. It may be related to the inflaton field rather simply as 

\begin{equation}
\mathcal{R} = -\frac{\mathcal{H}}{\bphi'}\delta\phi
\end{equation}

and so 

\begin{align}
\Delta^2_{\mathcal{R}}(k) &=(\frac{\mathcal{H}}{\bphi'})^2\Delta^2_{\delta\phi}(k)\\
&=\frac{1}{2\epsilon\Mp^2}(\frac{H}{2\pi})^2\rvert_{k=aH}
\end{align}

by \ref{epsilon}


........



 In principle we should consider the full action of the matter sector and the Einstein Hilbert term, though it turns out in spatially flat gauge the metric pertubations are slow roll supressed by a factor of $\epsilon$, and so we may consider only the matter sector. The Einstein-Hilbert term will become important later when we consider tensor perturbations to the metric.\\




\subsection{Gravitational Wave Background}

The remaining piece of the most general metric pertubation we must still consider are the two degrees of freedom in tensor pertubations. Considering just these we have

\begin{equation}
ds^2 = a^2(\tau)(-d\tau^2 + (\delta_{ij}+\gamma_{ij})dx^idx^j)
\end{equation}

where $\gamma_ij$ is symmetric, transverse, and traceless: ie $\gamma_{ij}=\gamma_{ji}$, $\partial_i\gamma_{ij}=0$ and $\gamma_{ii}=0$.

As previously, we need to insert this into the action, which here is the Einstein-Hilbert action, and expand to second order in pertubations. This calculation is rather long, so we omit the details here, but one can find them at gjkdhfghsdfg. 

\begin{equation}
S=\frac{\Mp^2}{2} \int d\tau d^3x \sqrt{-g}R \rightarrow S^{(2)} = \frac{\Mp^2}{8}\int d\tau d^3x \gamma_{ij}'\gamma_{ij}'-\partial_k\gamma_{ij}\partial_k\gamma_{ij}
\end{equation}

By the symmetries of the problem these are the only terms we expect to appear at second order, though one has to go through the calculation to get the correct numerical factors as they turn out to be important. We may expand the graviton in plane waves 

\begin{equation}
\gamma_{ij}(\tau, \v{x}) = \fint{k} \sum_{s=+,x} \epsilon_{ij}^s(\v{k})\gamma_s(\tau,\v{k})e^{i\v{k}\cdot\v{x}}
\end{equation}

where $\epsilon_{ij}^s$ are in general complex polarisation tensors satisfying


\begin{align}
\epsilon_{ii}^s(\v{k}) &= k^i \epsilon_{ij}^s(\v{k}) = 0 &\text{transverse and traceless}\\
\epsilon_{ij}^s(\v{k}) &= \epsilon_{ji}^s(\v{k}) &\text{symettric}\\
\epsilon_{ij}^s(\v{k})\epsilon_{jk}^s(\v{k}) &= 0&\text{null}\\
\epsilon_{ij}^s(\v{k})\epsilon_{ij}^s(\v{k})^* &= 2\delta_{ss'} &\text{normalisation}\\
\epsilon_{ij}^s(\v{k})^* &= \epsilon_{ij}^s(\v{-k})&\text{$\gamma_{ij}$ real}
\end{align}

inserting this expansion into $S^{(2)}$ we obtain 

\begin{equation}
S^{(2)} = \frac{\Mp^2}{2} \half \fint{k} d\tau a^2 \sum_{s=+,x} \gamma_s'(\tau,\v{k})\gamma_s '(\tau,\v{-k})+k^2 \gamma_s(\tau,\v{k})\gamma_s (\tau,\v{-k})
\label{gravwaveaction}
\end{equation}

We find this is, up to a constant, really just two copies of a special case of the action \ref{scalarfieldaction}. To see this, consider write \ref{scalarfieldaction} in fourier space $\phi = \fint{k} \phi(\v{k})e^{i\v{k}\cdot\v{x}}$, in the special case where $\bphi = V(\bphi) = 0$, i.e. $\phi = \delta \phi$.


\begin{align}
S &= \half \int d\tau d^3x a^2 [(\phi ' )^2 -(\nabla \phi)^2]\\
 &= \half \int d\tau d^3x a^2 [\phi'(\v{k})\phi'(\v{-k}) + k^2 \phi(\v{k})\phi(\v{-k})]
\end{align}

we see \ref{gravwaveaction} is just two independent copies of this action, allowing us to quantise the two independent fields $\tilde{\gamma_s} = \frac{\Mp a}{\sqrt{2}} \gamma_s$ exactly as we did $ f = a \delta \phi$. The normalisation is required to give canonical factor of $\half$ on the kinetic term in the action, which plays a role when we fix the Wronskian to 1. Going through the same procedure as before, we get:

\begin{itemize}
\item{operators $\hat{\gamma}_s(\v{k}) = \frac{\sqrt{2}\Mp}{a}(f_k\anns{k}{s}+f_k^*\cres{k}{s})$}
\item{commutation relations $[\anns{k}{s},\cres{k'}{s}] = (2\pi)^3\delta(\v{k}-\v{k'})\delta_{ss'}$}
\item{the same bunch-davies mode functions $f_k$}
\end{itemize}

The final result of this section is to calculate the tensor power spectrum. 
We have

\begin{align}
\langle\gamma_{ij}(\v{k})\gamma_{ij}(\v{k'})\rangle & = \sum_{ss'} \epsilon^s_{ij}(\v{k})\epsilon^s_{ij}(\v{k'})\langle\gamma_{s}(\v{k})\gamma_{s}(\v{k'})\rangle\\
&= (\frac{\sqrt{2}\Mp}{a})^2 \sum_{ss'} \epsilon^s_{ij}(\v{k})\epsilon^s_{ij}(\v{k'})(2\pi)^3\delta(\v{k}+\v{k'})|f_k|^2\\
&= (\frac{\sqrt{2}\Mp}{a})^2 \sum_{ss'} 2\delta_{ss'}(2\pi)^3\delta(\v{k}+\v{k'})|f_k|^2\\
&= (2\pi)^3\delta(\v{k}+\v{k'})\frac{8}{\Mp^2a^2}|f_k|^2
\end{align}

using several of the properties listed earlier. We can read off 

\begin{equation}
P_t(k)=\frac{8}{\Mp^2}P_{\delta\phi}(k) \Rightarrow \Delta^2_t(k)=\frac{8}{\Mp^2}\Delta^2_{\delta\phi}(k) = \frac{2}{\pi^2}(\frac{H}{\Mp})^2
\end{equation}


\subsection{Properties of the scalar and tensor power spectra}

In summary, we have calculated the scalar and tensor power spectra:

\begin{equation}
\Delta^2_{s}(k) = \frac{1}{2\epsilon\Mp^2}(\frac{H}{2\pi})^2\rvert_{k=aH} \qquad
\Delta^2_t(k)= \frac{2}{\pi^2}(\frac{H}{\Mp})^2\rvert_{k=aH}
\end{equation}

These freeze out after horizon exit, and thus provide the initial conditions of the universe. Given a good understanding of the physics of proceeding cosmological evolution we have several direct probes of these power spectra. The remainder of this essay will go into more detail about how we do this this: we will explicitly calculate the effect these scalar and tensor perturbations have on several CMB observables, thus providing a way to detect these primordial fluctuations.\\

Here we investigate some of their properties. Recall H is approximately constant during inflation, but not perfectly so, since $H^2 \sim V(\phi)$, and $\phi$ is slowly rolling down its potential. Similarly $\epsilon$ also varies slightly. We can parametrise this deviation from scale invariance to first order by the scalar and tensor spectral indexes $n_i$, and also introduce the respective amplitudes $A_i$ via:

\begin{equation}
\Delta^2_{\mathcal{R}}(k) \approx A_s(\frac{k}{k_*})^{n_s-1} \qquad
\Delta^2_t(k) \approx A_t(\frac{k}{k_*})^{n_t}
\end{equation}

where $k_*$ is some pivot scale, often taken to be $0.05Mpc^{-1}$.

We can compute the spectral indexes in terms of the slow roll parameters. 

\begin{align}
n_s-1 &= \frac{d\ln{\Delta^2_{\mathcal{R}}}}{d\ln{k}} = \frac{d}{d\ln{k}}(2\ln{H}-ln{\epsilon} \approx \frac{1}{H}\frac{d}{dt}(2\ln{H}-\ln{\epsilon}) = \frac{2\dot{H}}{H^2}-\frac{\dot{\epsilon}}{H\epsilon} = -2\epsilon-\eta = -6\epsilon_V+2\eta_V\\
n_t &= \frac{d\ln{\Delta^2_t}}{d\ln{k}} = \frac{d}{d\ln{k}}2\ln{H}=-2\epsilon=-2\epsilon_V
\end{align}

noting that that $d\ln{k} = d\ln{aH} \approx d\ln{a} = Hdt$ since we evaluate the power spectra 
at horizon exit, and H is slowly varying (so $d\ln{H}$ provides a next to leading order correction.


During inflation $\epsilon>0$ since energy density is monotonically decreasing, and so $n_t<0$ and tensor pertubations are said to have a red spectrum.

We can also define the scalar to tensor ratio 

\begin{equation}
r=\frac{A_t}{A_s} \approx 16\epsilon
\end{equation}

leading to the consistency condition $r=-8n_t$, which serves as a test, at least in principle, of SFSR. Other inflation models predict different consistency conditions.


\subsection{What can we learn?}
We can learn several interesting things about the physics of inflation itself from these. We'll go into some detail about several. It shouldn't be suprising that this information is contained within the parameters we previously defined. Writing $n_s$ in terms of the potential slow roll parameters indicates $n_s$ contains information about the shape of the inflationary potential.\\ 

Recent CMB experiments have managed to measure $n_s$ to good precision to be $n_s=0.968\pm0.006$, differing from a scale invariant $n_s=1$ by over $5\sigma$. [REF] SFSR inflation models can be written down with either a red ($n_s<1$) or blue ($n_s>1$ spectrum, and this measurement therefore rules out a large slew of such models. \\

At present, we have gained no statistically significant evidence for tensor modes, i.e. have no measured value of $n_t$ or $r$, though the lack of detection constrains $r<0.1$ [BICEP 2 KECK + PLANCK]. Upcoming experiments [CMbS4] seek to improve this, though as we will show this constraint is already useful. 

Given the large number of inflation models, even within the SFSR regime, that exist in the literature, it is useful to work model independently. We'll describe some ifnromation we can gather:

\subsubsection{The energy scale of inflation}

In the slow roll approximation, we may write 

\begin{equation}
\Delta^2_{\mathcal{R}}\approx \frac{V}{24\pi^2\epsilon\Mp^4}
\end{equation}

Current measurements [QBM -> PLANCK] give $\Delta^2_{\mathcal{R}} \approx A_s \approx 2.2\times10^{-9}$. Making use of $r\approx 16\epsilon$, we learn

\begin{equation}
V=24\pi^2A_s\frac{r}{16}\Mp^4 \Rightarrow V^{1/4} \approx 3.12\times10^{16}\text{GeV}r^{1/4} \leq 1.75\times10^{16}\text{GeV}
\end{equation}

This can be further improved with more accurate bounds on r.

\subsubsection{Field excursion}
Recall \ref{efolds} giving 

\begin{equation}
N(t) =  \int_{\bphi(t)}^{\bphi_end} \frac{d\bphi}{\sqrt{2\epsilon_V}\Mp}
\Rightarrow dN=\frac{d\bphi}{\sqrt{2\epsilon_V}\Mp}
\Rightarrow \frac{\Delta \phi}{\Mp} = \int_0^N \sqrt{2\epsilon} dN = \int_0^N \sqrt{r/8} dN 
\end{equation}



























\section{CMB}

\subsection{Introduction}

\subsection{Observables}
A CMB Photon detector measures the electric field $\v{E}$ perpendicular to the direction of observation $\hat{\v{n}}$, from which we can definite a rank two intensity correlation tensor

\begin{equation}
I_ij = \langle E_aE_b* \rangle
\end{equation}

where $\langle\rangle$ denotes a time average over many periods. Rank two tensors can be decomposed into three irreducible components: a part proportional to $\delta_{ab}$, a symettric tracefree part, and an antisymettric part. The antisymettric part encodes a phase lag between $E_1$ and $E_2$, and therefore circular polarization. Thomson scattering induces no such polarisation \footnote{as shown for example in [Kowosky], since the V boltzmann equation has no source term}, and so we may neglect this. Fixing an orthonormal basis $\{\unit{e_1},\unit{e_2}\}$ we encode the three remaining degrees of freedom as 

\begin{equation}
I_{ij} = T\delta{ij}+2P_ij
=\begin{pmatrix}
T+Q & U\\ 
U & T-Q
\end{pmatrix}
\end{equation}

where 

\begin{equation}
P_{ij} =\half \begin{pmatrix}
Q & U\\ 
U & -Q
\end{pmatrix}
\end{equation}

is the polarisation tensor and Q and U are standard Stokes' parameters. Note $P_{ij}$ has eigenvalues $\pm (Q^2 + U^2)^\half$, and eigenvectors making an angle of $\half\arctan(\frac{U}{Q})$ with $\unit{e_1}$. We call the magnitude of the eigenvalue the polarisation amplitude. Diagramtically, we often depict polarisation as headless vectors of length $(Q^2 + U^2)^\half$, making an angle of $\arctan(\frac{U}{Q})$ to $\unit{e_1}$, which indicates the direction of measurement which would maximise the signal.


Since $\delta_{ij}$ is an invariant tensor under rotations we see that $T$ is invariant under rotations, but $Q$, and $U$ are not. In particular if we perform a right handed rotation around $\unit{n}$ by an angle $\psi$ we get the following transformations:


\begin{equation}
\begin{pmatrix}
\unit{e_1}'\\
\unit{e_2}' 
\end{pmatrix}
=
\begin{pmatrix}
\cos{\psi} & \sin{\psi}\\ 
\-sin{\psi} & \cos{\psi}
\end{pmatrix}
\begin{pmatrix}
\unit{e_1}\\
\unit{e_2} 
\end{pmatrix}
\end{equation}


\begin{equation}
\begin{pmatrix}
Q'\\
U' 
\end{pmatrix}
=
\begin{pmatrix}
\cos{2\psi} & \sin{2\psi}\\ 
\-sin{2\psi} & \cos{2\psi}
\end{pmatrix}
\begin{pmatrix}
Q\\
U
\end{pmatrix}
\label{QUtranform}
\end{equation}

The fact that Q and U are basis dependent quantities means they aren't physical: we will fix this soon.

MAYBE DROP
An alternative choice of basis is given by the complex combinations $e_\pm = \unit{e_1} \pm i\unit{e_2}$, with respect to which the components of $P$ are 



It is convinient to instead work with quantities $Q\pm iU$. Using \ref{QUtranform} it is easy to see under a rotation we get  

\begin{equation}
Q'\pm iU' = e^{-2i\psi}(Q\pm iU)
\end{equation}

A function $f(\theta, \phi)$ defined on a sphere is said to have spin $s$ if under a right handed rotation by angle $\psi$ of orthogonal vectors tangential to the sphere $(\unit{e_1}, \unit{e_2})$, transforms as $f'(\theta, \phi) = e^{-is\psi}f(\theta, \phi)$. Thus $Q+iU$ is a spin 2 quantity, and $Q-iU$ a spin 2 quantity. From here onwards we use the natural choice of tangential basis on the sphere $(\unit{e_1}, \unit{e_2}) = (\unit{e_theta}, \unit{e_\phi})$. The reason we construct these quantities of definite spin, is that analogously to the expansion of a scalar quantity

\begin{equation}
T(\unit{n}) = \sum_{lm} T_{lm}Y_{lm}(\unit{n})
\end{equation}

in terms of spherical harmonics, there exist so called `spin weighted spherical harmonics' $_sY_{lm}$ forming a complete orthonormal basis for spin s weighted functions:

\begin{equation}
(Q\pm iU)(\unit{n}) = \sum_{lm} a_{\pm2,lm} {}_{\pm2}Y_{lm}(\unit{n})
\label{QUexp}
\end{equation}

By reality conditions on $T, Q$ and $U$, we have

\begin{equation}
T_{lm}^* = T_{l,-m} \qquad a_{-2lm}^*=a_{2lm}
\end{equation}

Now Q and U are defined with respect to different bases all over the sphere and depend on this choice of basis, so cannot be compared at different positions in a full sky treatment. We amend this issue by using spin raising and lowering operators to construct two rotationally invariant quantities out of $Q\pm iU$. Acting twice on \ref{QUexp} with our raising and lowering operators we get

\begin{align}
\sr^2(Q-iU)(\unit{n}) &= \sum_{lm} \ltwo^{1/2} a_{2,lm} {}Y_{lm}(\unit{n})\\
\sl^2(Q+iU)(\unit{n}) &= \sum_{lm} \ltwo^{1/2} a_{-2,lm} {}Y_{lm}(\unit{n})
\end{align}

where now we can express $a_{\pm2,lm}$ using orthogonality in two ways


\begin{align}
a_{2,lm} &= \int d\Omega {}_2Y_{lm}^*(\unit{n})(Q+iU)(\unit{n})\\
&= \ltwo^{-1/2}\int d\Omega Y_{lm}^*(\unit{n})\sl^2(Q+iU)(\unit{n})\\
a_{-2,lm} &= \int d\Omega {}_2Y_{lm}^*(\unit{n})(Q-iU)(\unit{n})\\
&= \ltwo^{-1/2}\int d\Omega Y_{lm}^*(\unit{n})\sr^2(Q-iU)(\unit{n})
\end{align}

We now introduce two linear combinations, expressible in terms of the rotationally invariant quantities, which will be slightly more physical. Note there is a choice of sign made here which ultimately won't matter

\begin{align} 
E_{lm} &= -(a_{2,lm} + a_{-2,lm})/2\\
B_{lm} &= -(a_{2,lm} - a_{-2,lm})/2
\end{align}

Using these we can rewrite \ref{QUexp} as 

\begin{equation}
(Q\pm iU)(\unit{n}) = -\sum_{lm} (E_{lm} \pm i B_{lm}) {}_{\pm2}Y_{lm}(\unit{n})
\label{QUEB}
\end{equation}

In real space we introduce related quantities 

\begin{align}
\tilde{E}(\unit{n}) &:= -\half[\sl^2(Q+iU) + \sr^2(Q-iU)]\\
&= \sum_{lm} \ltwo^{1/2}E_{lm}Y_{lm}(\unit{n})\\
\tilde{B}(\unit{n}) &:= i\half[\sl^2(Q+iU) - \sr^2(Q-iU)]\\
&= \sum_{lm} \ltwo^{1/2}B_{lm}Y_{lm}(\unit{n})
\label{EBmodes}
\end{align}

which only differs from how you'd expect to define $E$ and $B$ by factors of $\ltwo$ on each multipole: 

\begin{align}
\tilde{E}_{lm}&=\ltwo^{1/2}E_{lm}
\tilde{B}_{lm}&=\ltwo^{1/2}B_{lm}
\label{EBtwiddle}
\end{align}

\subsubsection{Parity}

Note that all the $E$ and $B$ quantities we have defined are rotationally invariant, and therefore carry some physical meaning. In this short section we attempt to gain some intuition to what that is.

We will first show E and B have distinct parity. Consider a space inversion reversing the sign of the x coordinate. In spherical coordinates this leaves $r, \theta$ invariant, but varies $\phi \rightarrow -\phi$.  Let $\unit{n}=(\theta, \phi)$ and $\unit{n}'=(\theta', \phi')$ refer to the same physical direction in the two frames. From the definition of the Stokes parameters $Q=\langle |E_x|^2 \rangle - \langle |E_y|^2 \rangle$ and $U = \langle E_xE_y^* \rangle + \langle E_yE_x^* \rangle$ we expect $Q'(\unit{n}') = Q'(\unit{n})$ but $U'(\unit{n}') = -U(\unit{n})$. Thus $(Q\pm iU)'(\unit{n}') = (Q\mp iU)(\unit{n})$. Now we should act with spin raising and lowering operators twice. We have

\begin{align}
\sl(Q+iU)'(\unit{n}') &= -\sin^{-2}{\theta'}[\partial_{\theta'}-i\csc{\theta'}\partial_{\phi'}]\sin^{2}{\theta'}(Q+iU)'(\unit{n}')\\
&= -\sin^{-2}{\theta}[\partial_{\theta}+i\csc{\theta}\partial_{\phi}]\sin^{2}{\theta}(Q-iU)'(\unit{n})\\
&= \sr (Q-iU)(\unit{n})
\end{align}

and repeating almost the same calculation gives 

\begin{align}
\sl^2(Q+iU)'(\unit{n}') = \sr^2 (Q-iU)(\unit{n})\\
\sr^2(Q-iU)'(\unit{n}') = \sl^2 (Q+iU)(\unit{n})
\end{align}

and so we see from \ref{EBmodes} that under parity $\tilde{E}'(\unit{n'})=\tilde{E}(\unit{n})$ but $\tilde{B}'(\unit{n'})=-\tilde{B}(\unit{n})$. This parity property is the important distinguishing feature of E and B modes, and can be used to understand what a typical E and B mode pattern looks like:

TO ADD

\subsection{Statistics}
Parity also constrains the statistics of CMB pertubations. From measurements, we get the expansion coefficients $T_{lm}. E_{lm}$ and $B_{lm}$. We care about the cross correlations between these various quantities. There are four rotationally invariant power spectra of interest:

\begin{align}
\langle T_{lm}T_{l'm'} \rangle &= C_{Tl}\delta_{ll'}\delta_{mm'}
\langle E_{lm}E_{l'm'} \rangle &= C_{El}\delta_{ll'}\delta_{mm'}
\langle B_{lm}B_{l'm'} \rangle &= C_{Bl}\delta_{ll'}\delta_{mm'}
\langle T_{lm}E_{l'm'} \rangle &= C_{Cl}\delta_{ll'}\delta_{mm'}
\end{align}

since by parity, the cross correlations $B$ with $T$ and $E$ must vanish. We will show this later explicitly. 





\subsection{The flat sky approximation}

The flat sky approximation is the small scale limit of the above discussion. It neglects the curvature of the sphere, allowing us to work instead with a standard flat space fourier basis, instead of (spin-weighted) spherical harmonics. It provides a useful tool to give some intuition about physical effects, though to link to observation a full sky treatment is required, especially given we expect for example the B mode polarisation to only contribute on large angular scales. It is also historically relevant: much of the early work on CMB polarisation focussed on this limit, before the techniques discussed above to deal with the full sky were discovered. Here, we'll describe how to get the small scale limit out of the full sky treatment.

We take $\unit{n}$ to be close to $\unit{z}$, and substitute as follows, bearing in mind we are working with large l:

\begin{equation}
T(\unit{n}) = \sum_{lm} T_{lm}Y_{lm}(\unit{n}) \rightarrow (2\pi)^{-2} \int d^2\v{l} T(\v{l})e^{i\v{l}\cdot\unit{n}}
\end{equation}

\begin{align}
{}_2Y_{lm} &= \ltwo^\half\sr^2Y_{lm} \rightarrow (2\pi)^{-2} \frac{1}{l^2}\sr^2e^{i\vec{l}\cdot\unit{n}}\\
{}_{-2}Y_{lm} &= \ltwo^\half\sl^2Y_{lm} \rightarrow (2\pi)^{-2} \frac{1}{l^2}\sl^2e^{i\vec{l}\cdot\unit{n}}
\end{align}

and so \ref{QUEB} becomes

\begin{align}
(Q+iU)(\unit{n}) &= -(2\pi)^2\int d^2l [E(l)+iB(l)]  \frac{1}{l^2}\sr^2e^{i\vec{l}\cdot\unit{n}}\\
(Q+iU)(\unit{n}) &= -(2\pi)^2\int d^2l [E(l)+iB(l)] \frac{1}{l^2}\sl^2e^{i\vec{l}\cdot\unit{n}}
\end{align}

In the small scale limit we can derive:

\subsection{The Boltzmann Equation}

We use kinetic theory to describe the evolution of photons from the fluid-like, pre-recombination regime to free-streaming radiation. To track pol


\subsection{Power spectrum of tensor modes}

Now we have set up all the machinery we require in order to calculate polarisation correlations on the sphere. The main goal of this section is to show that tensor pertubations source a non zero B mode polarisation, whose power spectrum depends on the primordial tensor power spectrum. Our starting point is the boltzmann equation describing the evolution of temperature and polarisation fluctuations.  We'll work in Fourier space throughout, focussing on a mode with wave vector $\unit{k}$, which we may choose to be aligned with the $\unit{z}$ direction, and take the basis on which we evaluate $Q$ and $U$ to be the natural basis on the sphere: $(\unit{e_1}, \unit{e_2}) = (\unit{e_\theta}, \unit{e_\phi})$.

As we saw earlier, there exist two independent polarizations $\gamma_s$ of the gravitational wave in fourier space: $+$ and $\times$. We work instead with the following linear combinations, and change notation to use $\zeta$ instead

\begin{equation}
\zeta^1 = (\zeta^+ - i\zeta^\times)/\sqrt{2} \qquad \zeta^2 = (\zeta^+ + i\zeta^\times)/\sqrt{2}
\end{equation}

These modes are independent random variables, each having half of the primordial tensor power spectra:

\begin{equation}
\langle \zeta^{1*}(\v{k})\zeta^{1*}(\v{k'})\rangle=\langle \zeta^{2*}(\v{k})\zeta^{2*}(\v{k'})\rangle=\half P_t(k)\delta(\v{k}-\v{k'})
\end{equation}
\begin{equation}
\langle \zeta^{1*}(\v{k})\zeta^{2*}(\v{k'})\rangle=0
\end{equation}

In order to decouple the Boltzmann equations describing evolution of T, U and Q, we introduce new variables, first introduced by Polnarev [] $\tilde{\Delta}_T$ and $\tilde{\Delta}_P$ describing the temperature and polarisation pertubations induced by gravitational waves. They are written in terms of the standard pertubations as:

\begin{align}
\Delta_T(\tau,\unit{n},\v{k}) &= [(1-\mu^2) e^{2i\phi} \zeta^1(\v{k})+(1-\mu^2) e^{-2i\phi} \zeta^2(\v{k})]\tilde{\Delta_T}(\tau,\mu,k)\\
\Delta_Q+i\Delta_U((tau,\unit{n},\v{k}) &=[(1-\mu)^2 e^{2i\phi} \zeta^1(\v{k})+(1+\mu)^2 e^{-2i\phi} \zeta^2(\v{k})]\tilde{\Delta_P}(\tau,\mu,k) \\
\Delta_Q-i\Delta_U((tau,\unit{n},\v{k}) &=[(1+\mu)^2 e^{2i\phi} \zeta^1(\v{k})+(1-\mu)^2 e^{-2i\phi} \zeta^2(\v{k})]\tilde{\Delta_P}(\tau,\mu,k)
\end{align}

and satisfy the two uncoupled Boltzmann equations

\begin{align}
\tilde{\Delta}_T'+ik\mu \tilde{\Delta}_T &= -h' -\kappa'[\tilde{\Delta}_T - \Psi]\\
\tilde{\Delta}_P'+ik\mu\tilde{\Delta}_P &= -\kappa'[\tilde{\Delta}_P + \Psi]
\end{align}

where the source is

\begin{equation}
\Psi = \frac{1}{10}\tilde{\Delta}_{T0} + \frac{1}{7}\tilde{\Delta}_{T2} + \frac{3}{70}\tilde{\Delta}_{T4} - \frac{3}{5}\tilde{\Delta}_{P0} + \frac{6}{7}\tilde{\Delta}_{P2} - \frac{3}{70}\tilde{\Delta}_{P4}
\end{equation}


involving multipole moments, defined for any function as $f(k, \mu) = \sum_l (2l+1)(-i)^lf_l(k)P_l(\mu)$ for $P_l(\mu)$ the order l legendre polynomial. All derivatives are taken with respect to conformal time $\tau$. The \textit{differential optical depth} for Thomson scattering is $\kappa'=an_ex_e\sigma_T$, where $n_e$ is the electron density, $x_e$ is the ionization fraction, and $\sigma_T$ is the Thomson cross section. The \textit{optical depth} is given by the integral $\kappa(\tau) = \int_\tau^{\tau_0} \kappa'(tau)d\tau$. 


We may now formally integrate these first order linear ODEs, using an `integrating factor' - this is the `line of sight' method. Introducing $x=k(\tau_0-\tau)$ we get that 

\begin{align}
\tilde{\Delta}_T(\tau,\mu,k) &= \int_0^{\tau_0} e^{ix\mu}S_T(k,\tau)\\
\tilde{\Delta}_P(\tau,\mu,k) &= \int_0^{\tau_0} e^{ix\mu}S_P(k,\tau)
\end{align}

where all the physics is hidden in the source terms:

\begin{align}
S_T(k,\tau) &= -h'e^{-\kappa}+g\Psi\\
S_P(k,\tau) &= -g\Psi
\end{align}

$g(\tau) = \kappa'e^{-\kappa}$ is the visibility function, peaking at the epoch of recombination. A simple approximation takes recombination to be instantaneous, at $\tau_*$, in which case $g(\tau) = \delta(\tau-\tau_*)$. Combining the last two results we attain expressions for the original T,Q,U pertubations we care about:



\begin{align}
\Delta_T(\tau,\unit{n},\v{k}) &= [(1-\mu^2) e^{2i\phi} \zeta^1(\v{k})+(1-\mu^2) e^{-2i\phi} \zeta^2(\v{k})]\int_0^{\tau_0} e^{ix\mu}S_T(k,\tau)\\
(\Delta_Q+i\Delta_U)(\tau,\unit{n},\v{k}) &=[(1-\mu)^2 e^{2i\phi} \zeta^1(\v{k})+(1+\mu)^2 e^{-2i\phi} \zeta^2(\v{k})]\int_0^{\tau_0} e^{ix\mu}S_P(k,\tau)\\
(\Delta_Q-i\Delta_U)(\tau,\unit{n},\v{k}) &=[(1+\mu)^2 e^{2i\phi} \zeta^1(\v{k})+(1-\mu)^2 e^{-2i\phi} \zeta^2(\v{k})]\int_0^{\tau_0} e^{ix\mu}S_P(k,\tau)
\end{align}



The T result is what we want. For Q and U however we saw earlier these are coordinate dependent quantities on the sphere, and so instead we work with $\tilde{E}$ and $\tilde{B}$. To do so, by \ref{EBmodes} all we need to do is act twice via our raising and lowering operators. Here is where we need \ref{+-2}, which conveniently pass through the source terms, which have no angular dependence. We get

\begin{align}
\sl^2(\Delta_Q+i\Delta_U)(\tau,\unit{n},\v{k}) &= \zeta^1(\v{k})e^{2i\phi}\int_0^{\tau_0}S_P(k,\tau)(\partial_\mu+\frac{2}{1-\mu^2})^2[(1-\mu^2)(1-\mu)^2 e^{ix\mu}]\\
&{} + \zeta^2(\v{k})e^{-2i\phi}\int_0^{\tau_0}S_P(k,\tau)(\partial_\mu+\frac{2}{1-\mu^2})^2[(1-\mu^2)(1+\mu)^2 e^{ix\mu}]\\
&= \zeta^1(\v{k})e^{2i\phi}\int_0^{\tau_0}S_P(k,\tau)(-\mathcal{E}(x)-i\mathcal{B}(x))[(1-\mu^2) e^{ix\mu}] \\
&{} + \zeta^2(\v{k})e^{-2i\phi}\int_0^{\tau_0}S_P(k,\tau)(-\mathcal{E}(x)-i\mathcal{B}(x))[(1-\mu^2) e^{ix\mu}] not sure about this line
\end{align}

\begin{align}
\sr^2(\Delta_Q-i\Delta_U)(\tau,\unit{n},\v{k}) &= \zeta^1(\v{k})e^{2i\phi}\int_0^{\tau_0}S_P(k,\tau)(\partial_\mu-\frac{2}{1-\mu^2})^2[(1-\mu^2)(1+\mu)^2 e^{ix\mu}] \\ 
&{} + \zeta^2(\v{k})e^{-2i\phi}\int_0^{\tau_0}S_P(k,\tau)(\partial_\mu-\frac{2}{1-\mu^2})^2[(1-\mu^2)(1-\mu)^2 e^{ix\mu}]\\
&= \zeta^1(\v{k})e^{2i\phi}\int_0^{\tau_0}S_P(k,\tau)(-\mathcal{E}(x)+i\mathcal{B}(x))[(1-\mu^2) e^{ix\mu}] \\
&{} + \zeta^2(\v{k})e^{-2i\phi}\int_0^{\tau_0}S_P(k,\tau)(-\mathcal{E}(x)+i\mathcal{B}(x))[(1-\mu^2) e^{ix\mu}] not sure about this line
\end{align}
where we have introduced differential operators, which will save us some work very soon.

\begin{equation}
\mathcal{E}(x)=-12+x^2(1-\partial^2_x)-8x\partial_x \qquad \mathcal{B}(x) = 8x+2x^2\partial_x
\end{equation}

Plugging these into \ref{EBmodes} we find our final results for each fourier mode to be:

\begin{align}
\Delta_T(\tau_0,\unit{n},\v{k}) &= [(1-\mu^2) e^{2i\phi} \zeta^1(\v{k})+(1-\mu^2) e^{-2i\phi} \zeta^2(\v{k})]\int_0^{\tau_0} e^{ix\mu}S_T(k,\tau)\\
\Delta_{\tilde{E}}(\tau_0,\unit{n},\v{k}) &= [(1-\mu^2) e^{2i\phi} \zeta^1(\v{k})+(1-\mu^2) e^{-2i\phi} \zeta^2(\v{k})]\mathcal{E}(x)\int_0^{\tau_0} e^{ix\mu}S_P(k,\tau)\\
\Delta_{\tilde{B}}(\tau_0,\unit{n},\v{k}) &= [(1-\mu^2) e^{2i\phi} \zeta^1(\v{k})+(1-\mu^2) e^{-2i\phi} \zeta^2(\v{k})]\mathcal{B}(x)\int_0^{\tau_0} e^{ix\mu}S_P(k,\tau)
\end{align}

The complete solution requires integration over all fourier modes, which evolve independently. Note the fourier factor is omitted since we measure each mode from the same spatial position $\v{x}$.

FOURIER CONVENTION NEEDS MODIFYING HERE: IJUSFHAIUSHF

\begin{align}
\Delta_T(\tau_0,\unit{n}) &= \int d^3k \Delta_T(\tau_0,\unit{n},\v{k})\\
\Delta_{\tilde{E}}(\tau_0,\unit{n}) &= \int d^3k\Delta_{\tilde{E}}(\tau_0,\unit{n},\v{k})\\
\Delta_{\tilde{B}}(\tau_0,\unit{n}) &= \int d^3k\Delta_{\tilde{B}}(\tau_0,\unit{n},\v{k})
\end{align}

Note here is where the convoluted definition of E and B modes help us out. Recall Q and U were defined with respect to a fixed basis, dependent on the the direction on the sphere $\unit{n}$. In order to add up all the modes over the sphere we would need to rotate each $Q\pm iU$ $\v{k}$ mode by a $\v{k}$ and $\unit{n}$ dependent phase(since we aligned $\v{k}$ with $\unit{z}$). This was a complication in early attempts to characterise the CMB polarisation beyond the flat sky regime, and we have avoided it through our definition of rotationally invariant E and B modes.\\

We see tensor perturbations have induced both E and B modes.

\subsection{Correlation Functions}

Since the form of these are so similar, we can compute the TT, EE and BB correlation functions in one fall swoop. We can also, with more work, verify that T does not correlate with E or B.

We'll begin by calculating the TT power spectrum ($C_{Tl}^{(T)}$, where the superscript reminds us these are only those sourced by tensor perturbations). Recalling the definition of the angular power spectrum, we first require the spherical multipole coeffiecents $T_{lm}$  

\begin{equation}
T_{lm} = \int d\Omega Y_{lm}^*(\unit{n})\Delta_T(\tau_0,\unit{n})
\end{equation}

From \ref{powerspectra} we can write 

\begin{align}
C_{Tl} &= \frac{1}{2l+1}\sum_m \langle T_{lm}^*T_{lm} \rangle \\
&= \frac{1}{2l+1} \sum_m [\int d\Omega Y_{lm}^*(\unit{n})\int d^3k\int_0^{\tau_0}d\tau S_T(\tau,k)e^{ix\mu}]^*
[\int d\Omega Y_{lm}^*(\tilde{\unit{n}})\int d^3\tilde{k}\int_0^{\tau_0}d\tilde{\tau}S_T(\tau,\tilde{k})e^{i\tilde{x}\tilde{\mu}}]\\
&\times
\langle [(1-\mu^2) e^{2i\phi} \zeta^1(\v{k})+(1-\mu^2) e^{-2i\phi} \zeta^2(\v{k})]^*[(1-\tilde{\mu}^2) e^{2i\tilde{\phi}} \zeta^1(\v{\tilde{k}})+(1-\tilde{\mu}^2) e^{-2i\tilde{\phi}} \zeta^2(\v{\tilde{k}})]
\end{align}

The expectation value evaluates to
SOMEHTING FISHY HERE
\begin{align}
&(1-\mu^2)(1-\tilde{\mu}^2)[\langle\zeta^{1*}(\v{k})\zeta^1(\v{\tilde{k}})e^{2i\tilde{\phi}-2i\phi}+\langle\zeta^{2*}(\v{k})\zeta^2(\v{\tilde{k}})e^{-2i\tilde{\phi}+2i\phi}]\\
&~ (1-\mu^2)(1-\tilde{\mu}^2)P_t(k)\delta(\v{k}-\v{\tilde{k}})e^{2i\tilde{\phi}-2i\phi}
\end{align}

where we've used $\phi\rightarrow -\phi$ 

\begin{align}
C_{Tl} = \frac{1}{2l+1} \int k^2 dk P_t(k) \sum_m \bigg| \int d\Omega Y^*_{lm}(\unit{n}) \int_0^{\tau_0} S_T(k,\tau)(1-\mu^2)e^{2i\phi}e^{ix\mu} \bigg|^2
\end{align}

To evaluate this, we use the expression $Y_{lm} = [\frac{(2l+1)(l-m)!}{4\pi(l+m)!}]^\half P_l^m(\mu)e^{-im\phi}$ for spherical harmonics in terms of legendre polynomials, and note that the $\phi$ integral gives 0 unless m=2, in which case it gives $2\pi$. So we get 

\begin{align}
C_{Tl} = 4\pi^2\ltwof \int k^2 dk P_t(k) \bigg|  \int_0^{\tau_0} S_T(k,\tau)\int_{-1}^1 d\mu P_l^2(\mu)(1-\mu^2)e^{ix\mu} \bigg|^2
\end{align}

To evaluate the $\mu$ integral, we use $P^m_l(\mu)=(-1)^m(1-\mu^2)^{m/2}(\partial_\mu)^mP_l(\mu)$


\begin{align}
\int d\mu P_l^2(\mu)(1-\mu^2)e^{ix\mu} \\
&= \int d\mu(1-\mu^2)^2 \partial_\mu^2P_l(\mu)e^{ix\mu}\\
&= \int d\mu \partial_\mu^2P_l(\mu)(1+\partial_x^2)^2e^{ix\mu}\\
&= \int d\mu P_l(\mu)(1+\partial_x^2)^2(x^2e^{ix\mu})\\
&= 2i^l(1+\partial_x^2)^2(x^2j_l(x))\\
&= 2i^l(\frac{j_l(x)}{x^2})
\end{align}

where we used a combination of and $\int d\Omega Y_{lm}^*(\unit{n})e^{ix\mu} = \sqrt{4\pi(2l+1)}i^l j_l(x) \delta_{m0}$ and $Y_{l0} = \sqrt{\frac{2l+1}{4\pi}}P_l(\mu)$. In the final line we used ODE for spherical bessel functions: $j_l''+\frac{2j_l'}{x}+(1-\frac{l(l+1)}{x^2}j_l)=0$.

Thus our final result is:

\begin{align}
C_{Tl} = (4\pi)^2\ltwof \int k^2 dk P_t(k) \bigg|  \int_0^{\tau_0} S_T(k,\tau)\frac{j_l(x)}{x^2} \bigg|^2
\end{align}

\subsection{E and B mode power spectra}

Now we abuse the similarity of expressions in dsfhs to write down the EE and BB power spectra. The angular dependence of $\Delta_{\tilde{E}}$ and $\Delta_{\tilde{B}}$ are exactly the same as those of $\Delta_T$. The expressions differ only in the $\mathcal{E}$ and $\mathcal{B}$ operators, acting seperately on each k mode, which can be applied after the angular integrals are performed. We lose the $l$ dependent factor in front by converting from $\tilde{E}, \tilde{B}$ to $E,B$, using \ref{EBtwiddle}


\begin{align}
C_{El} &= (4\pi)^2 \int k^2 dk P_t(k) \bigg|  \int_0^{\tau_0} S_T(k,\tau)\mathcal{E}(x)\frac{j_l(x)}{x^2} \bigg|^2\\
&= (4\pi)^2\int k^2 dk P_t(k) \bigg|  \int_0^{\tau_0} S_T(k,\tau)[-j_l(x) +j_l''(x)+\frac{2j_l(x)}{x^2} \frac{4j_l'(x)}{x}]\bigg|^2\\
C_{Bl} &= (4\pi)^2 \int k^2 dk P_t(k) \bigg|  \int_0^{\tau_0} S_T(k,\tau)\mathcal{B}(x)\frac{j_l(x)}{x^2} \bigg|^2\\
&= (4\pi)^2\int k^2 dk P_t(k) \bigg|  \int_0^{\tau_0} S_T(k,\tau)[2j_l'(x)+\frac{4j_l(x)}{x}]\bigg|^2
\end{align}


\subsection{Scalar pertubations}

One can go through an almost identical, and actually simpler, process to derive analgous power spectra sourced by scalar pertubations. We'll omit these here, but will explain the reason why they produce no B mode polarisation. As before, we work in fourier space with wave vector $\v{k}$, which we may choose to be aligned with the $\unit{z}$ direction, and take the basis on which we evaluate $Q$ and $U$ to be the natural basis on the sphere: $(\unit{e_1}, \unit{e_2}) = (\unit{e_\theta}, \unit{e_\phi})$. The angular dependence of the Boltzmann equations are only in $\mu=\v{k}\cdot\unit{n}$, and thus we have azimuthal symmetry, or $\phi$ independence, causing only the Q stokes parameter to be generated. Another way to say the same thing is that the polarisation eigenvector must be proportional to $\unit{e_\theta}$, which by the form of polarisation matrix forces $U=0$. Now by \ref{EBmodes}, we see if $U=0$, then $B(\unit{n})$ is identically 0. Physically speaking, the reason tensor perturbations induce both Q and U polarisation, is because of the additional degree of freedom we have in tensor modes compared to scalar modes.


\subsection{Summary}

We have seen that primordial metric tensor pertubations produce a B mode signal in the CMB polarisation, unlike scalar pertubations. Four angular power spectra depend on tensor pertubations, though three of these observables also depend also on scalar perturbations, since E and T are also sourced by  these. As we discussed in the inflation chapter, we know this is at least an order of magnitude larger a signal, likely more. Therefore B mode polarisation provides the most promising route to detecting primordial tensor pertubations. Note we have only so far discussed the early universe. As we will see in the next chapter, late time effects also generate a B mode signal, which for the purposes of studying inflation is a large is an obstacle, though some of these extra B modes provide interesting insights, useful elsewhere.














\section{Some examples and best-practices}
\label{sec:intro}

Here follow some examples of common features that you may wanto to use
or build upon.

For internal references use label-refs: see section~\ref{sec:intro}.
Bibliographic citations can be done with cite: refs.~\cite{a,b,c}.
When possible, align equations on the equal sign. The package
\texttt{amsmath} is already loaded. See \eqref{eq:x}.
\begin{equation}
\label{eq:x}
\begin{split}
x &= 1 \,,
\qquad
y = 2 \,,
\\
z &= 3 \,.
\end{split}
\end{equation}
Also, watch out for the punctuation at the end of the equations.


If you want some equations without the tag (number), please use the available
starred-environments. For example:
\begin{equation*}
x = 1
\end{equation*}

The amsmath package has many features. For example, you can use use
\texttt{subequations} environment:
\begin{subequations}\label{eq:y}
\begin{align}
\label{eq:y:1}
a & = 1
\\
\label{eq:y:2}
b & = 2
\end{align}
and it will continue to operate across the text also.
\begin{equation}
\label{eq:y:3}
c = 3
\end{equation}
\end{subequations}
The references will work as you'd expect: \eqref{eq:y:1},
\eqref{eq:y:2} and \eqref{eq:y:3} are all part of \eqref{eq:y}.

A similar solution is available for figures via the \texttt{subfigure}
package (not loaded by default and not shown here).
All figures and tables should be referenced in the text and should be
placed at the top of the page where they are first cited or in
subsequent pages. Positioning them in the source file
after the paragraph where you first reference them usually yield good
results. See figure~\ref{fig:i} and table~\ref{tab:i}.

\begin{figure}[tbp]
\centering 
\includegraphics[width=.45\textwidth,trim=0 380 0 200,clip]{example-image}
\hfill
\includegraphics[width=.45\textwidth,angle=180]{example-image}
\caption{\label{fig:i} Always give a caption.}
\end{figure}

\begin{table}[tbp]
\centering
\begin{tabular}{|lr|c|}
\hline
x&y&x and y\\
\hline
a & b & a and b\\
1 & 2 & 1 and 2\\
$\alpha$ & $\beta$ & $\alpha$ and $\beta$\\
\hline
\end{tabular}
\caption{\label{tab:i} We prefer to have borders around the tables.}
\end{table}

We discourage the use of inline figures (wrapfigure), as they may be
difficult to position if the page layout changes.

We suggest not to abbreviate: ``section'', ``appendix'', ``figure''
and ``table'', but ``eq.'' and ``ref.'' are welcome. Also, please do
not use \texttt{\textbackslash emph} or \texttt{\textbackslash it} for
latin abbreviaitons: i.e., et al., e.g., vs., etc.



\section{Sections}
\subsection{And subsequent}
\subsubsection{Sub-sections}
\paragraph{Up to paragraphs.} We find that having more levels usually
reduces the clarity of the article. Also, we strongly discourage the
use of non-numbered sections (e.g.~\texttt{\textbackslash
  subsubsection*}).  Please also see the use of
``\texttt{\textbackslash texorpdfstring\{\}\{\}}'' to avoid warnings
from the hyperref package when you have math in the section titles



\appendix

\section{Properties of spin-weighted functions}
We list here the properties of spin-weighted functions we'll need. 

For each s, there exist a set of spin-weighted functions $_sY_lm$ on the sphere, satisfying the same orthogonality and completeness relations as regular spherical harmonics

\begin{align}
\int_0^{2\pi} d\phi \int_{-1}^{1} d(\cos{\theta}) {}_sY_{l'm'}^*(\theta,\phi){}_sY_{lm}(\theta,\phi) = \delta_{ll'}\delta{mm'} \\
\sum_{lm} {}_sY_{lm}^*(\theta,\phi){}_sY_{lm}(\theta',\phi')=\delta(\phi-\phi')\delta(\theta-\theta')
\end{align}

There also exist spin raising ($\sr$) and spin lowering ($\sl$) operators, which raise or lower the spin weight of a function by 1. Their explicit expression is given by

\begin{align}
\sr {}_sf(\theta, \phi) &= -\sin^s{\theta}[\partial_\theta+i\csc{\theta}\partial_\phi]\sin^{-s}{\theta}{}_sf(\theta, \phi)\\
\sl {}_sf(\theta, \phi) &= -\sin^{-s}{\theta}[\partial_\theta-i\csc{\theta}\partial_\phi]\sin^{s}{\theta}{}_sf(\theta, \phi)
\end{align}

Polarisation quantities are spin $\pm 2$. In the case where we have ${}_{\pm2}f(\mu,\phi)$ satisfying $\partial_\phi{}_{\pm2}f=im_{\pm2}f$ we can derive the following forms of the $\sr^2$ and $\sl^2$ operators

\begin{align}
\sl^2{}_2f(\mu, \phi) = (-\partial_\mu+\frac{m}{1+\mu^2})^2[(1-\mu^2)_2f(\mu,\phi)]\\
\sr^2{}_{-2}f(\mu, \phi) = (-\partial_\mu-\frac{m}{1-\mu^2})^2[(1-\mu^2)_{-2}f(\mu,\phi)]
\label{+-2}
\end{align}


We can also relate $_sY_{lm}$ to $Y_lm$:

\begin{equation}
_sY_{lm} = 
\begin{cases}
[\frac{(l-s)!}{(l+s)!}]^{1/2}\sr^s Y_{lm} & 0\leq s\leq l\\
[\frac{(l-s)!}{(l+s)!}]^{1/2}(-1)^s\sl^s Y_{lm} & -l\leq s\leq 0
\end{cases}
\end{equation}


Finally the following properties are useful

\begin{align}
{}_sY_{lm}^* &= (-1)^s {}_{-s}Y_{l,-m}\\
\sr {}_sY_{lm} &= \sqrt{(l-s)(l+s+1)}{}_{s+1}Y_{l,m}\\
\sl {}_sY_{lm} &= \sqrt{(l-s)(l-s+1)}{}_{s-1}Y_{l,m}\\
\sl\sr {}_sY_{lm} &= -(l-s)(l+s+1){}_{s}Y_{l,m}
\label{propertiesofspin}
\end{align}

\section{Essay Description}
Our most promising theory for the early universe involves a phase of cosmic inflation, which not only rapidly expands and flattens the universe, but also generates the primordial density perturbations from quantum fluctuations in the inflaton field. While we have good evidence for inflation, e.g. from the Gaussianity, adiabaticity and near-scale invariance of the scalar density perturbations, one prediction of inflation has not yet been found: many inflationary models produce a stochastic background of primordial gravitational waves. A detection of this background would not only provide a definitive confirmation of inflation, but could also give new insights into the microphysics of inflation and, more broadly, physics at the highest energies.\\

The best current way of finding this gravitational wave background is to search for a characteristic pattern in the polarization of the Cosmic Microwave Background (CMB), the B-mode polarization. This essay should explain the physics underlying the search for this B-mode polarization pattern, which is currently a major area of research in cosmology.
The essay should first review the calculation of the gravitational wave background produced
by standard single-field slow-roll inflation, a standard result described in past Part III lecture notes as well as a comprehensive review of the field (Kamionkowski \& Kovetz 2016, henceforth KK16). The essay should also explain why the strength of the gravitational wave background(together with the scalar spectral index) can provide powerful constraints on the properties of inflation, such as the potential shape, energy scale, and field excursion (CMB-S4 2016, KK16).\\

Drawing on KK16, CMB-S4 2016, past lecture notes and other resources, the essay should provide a (brief) review of the basics of CMB polarization, describe what the CMB B-mode polarization is, and explain why it is a powerful probe of inflationary gravitational waves.\\

The remaining parts of the essay can, to some extent, be tailored to the student’s interests. One option is to explain in detail the major observational challenges in B-mode searches for inflationary gravitational waves, discussing the problems of foregrounds (Bicep/Keck/Planck 2015) and gravitational lensing as well as mitigation methods such as multifrequency cleaning and delensing (Smith et al. 2012). Another option is to focus more on the theoretical background, describing in detail different classes of inflationary models and what these generically predict for B-mode polarization (CMB-S4 2016 and references therein). Students may also discuss a combination of both observational and theoretical aspects.\\

\textbf{Relevant Courses}\\

\textit{Essential}: Cosmology\\

\textit{Useful}: Advanced Cosmology, Quantum Field Theory, General Relativity\\

\textbf{References}
\begin{enumerate}[label={[\arabic*]}]
\item {Kamionkowski, M. \& Kovetz, E. D. 2016, Annual Review of Astronomy and Astrophysics,
54, 227}
\item {CMB-S4 Science Book 2016, arXiv:1610.02743 (mainly chapter 2)}
\item {BICEP/Keck/Planck 2015, arXiv:1502.00612, Phys. Rev. Lett. 141 101301}
\item {Smith, K. M. et al. 2012, arXiv:1010.0048, JCAP, 06 014}
\item {Baumann, D., lecture notes: http://www.damtp.cam.ac.uk/user/db275/Cosmology/Lectures.pdf}
\end{enumerate}


\acknowledgments

This is the most common positions for acknowledgments. A macro is
available to maintain the same layout and spelling of the heading.

\paragraph{Note added.} This is also a good position for notes added
after the paper has been written.





% The bibliography will probably be heavily edited during typesetting.
% We'll parse it and, using the arxiv number or the journal data, will
% query inspire, trying to verify the data (this will probalby spot
% eventual typos) and retrive the document DOI and eventual errata.
% We however suggest to always provide author, title and journal data:
% in short all the informations that clearly identify a document.

\begin{thebibliography}{99}

\bibitem{a}
Author, \emph{Title}, \emph{J. Abbrev.} {\bf vol} (year) pg.

\bibitem{b}
Author, \emph{Title},
arxiv:1234.5678.

\bibitem{c}
Author, \emph{Title},
Publisher (year).


% Please avoid comments such as "For a review'', "For some examples",
% "and references therein" or move them in the text. In general,
% please leave only references in the bibliography and move all
% accessory text in footnotes.

% Also, please have only one work for each \bibitem.


\end{thebibliography}
\end{document}
