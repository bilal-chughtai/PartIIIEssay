\section{The Boltzmann Equation}

this section might have to go\\


Now that we understand how to describe the relevant quantities on the sphere, we must understand the physics describing their evolution in the early universe, at the epoch of recombination. We use the formalism of distribution functions, and the boltzmann equation, only slightly extending the treatment described in lectures. There are four distribution functions to keep track of now, one for each of the Stokes parameters. We'll make concrete the claim that we may neglect the V stokes parameter, by showing it doesn't evolve from its primordial black body $V=0$ distribution. We'll first derive the Liouville equation, the left hand side of the Boltzmann equation, describing the evolution of the phase space distribution without the presence of a collision term. We'll omit deriving the collision term for Thomson scattering in detail, as the calculation is rather long, but a full discussion may be found in [], on which much of this section is based, although some different conventions are used: notably we use conformal time, and have swapped the newtonian potentials.

We seek to understand both scalar and tensor pertubations to a background FLRW spacetime. We neglect vector pertubations, which decay and are unimportant, unless sourced continuously, by topological defects for example. We use the Newtonian gauge form of the perturbed metric, and work to linear order in pertubations throughout.

\begin{equation}
ds^2 = a^2(\tau) [-(1+2\Psi)d\tau^2 + ((1-2\Phi)\delta_{ij} +h_{ij})dx^idx^j]
\label{perturbmetric}
\end{equation}

where the newtonian potentials are $\Phi$ and $\Psi$ and the metric pertubations $h_{ij}$ are transverse and traceless. 


Photons are described by space time coordinate $x^\mu$ and four momentum $p^\mu=\frac{dx^\mu}{d\lambda}$, and are null

\begin{equation}\begin{split}
g_{\mu\nu}p^\mu p^\nu&=0\\
-a^2(1+2\Psi)((p^0)^2 + p^2) = 0\\
\Rightarrow p^0 = \frac{p}{a}(1-\Psi)
\end{split}\end{equation}

We have introduced $p^2=g_{ij}p^ip^j$, the physical photon momentum. We now factorise the spatial part of the metric into an amplitude an angular part $p^i = C\hat{p}^i$, such that $\delta_{ij}\hat{p}^i\hat{p}^j=1$ with $C$ being determines by the constraint:

\begin{equation}
a^2((1-2\Phi)\delta_{ij}+h_{ij})p^ip^j = p^2 \Rightarrow C=\frac{p}{a}(1+\Phi-\half h_ij\hat{p}^i\hat{p}^j)
\end{equation}

so the photon 4 momentum may be written as 

\begin{equation}
p^\mu = \frac{p}{a}(1-\Psi, (1+\Phi-\half h_{jk}\hat{p}^j\hat{p}^k)\hat{p}^i))
\end{equation}

The Liouville equation for a distribution $f=f(\tau, x^\mu, p^\mu)$ as

\begin{equation}
\frac{df}{d\tau} = \frac{\partial f}{\partial\tau} + \frac{\partial f}{\partial x^i} \frac{dx^i}{d\tau} + \frac{\partial f}{\partial p} \frac{dp}{d\tau} + \frac{\partial f}{partial\hat{p}^i} \frac{d\hat{p}^i}{d\tau}  = 0
\label{Liouville}
\end{equation}

We wish to evaluate this to 0'th and 1'st order. It can be shown the final term is second order, as both $\frac{\partial f}{partial\hat{p}^i}$ and  $\frac{d\hat{p}^i}{d\tau}$ are first order quantities. Physically, the first is because the background distribution is isotropic, and the second is since the unperturbed geodesics are straight lines. \\

The $\frac{dx^i}{d\tau}$ term is easily computed:

\begin{equation}
\frac{dx^i}{d\tau} = \frac{dx^i}{d\lambda}\frac{d\lambda}{d\tau} = \frac{p^i}{p^0} = (1+\Phi+\Psi-\half h_jk\hat{p}^j\hat{p}^k)\hat{p}^i
\end{equation}

whereas $\frac{dp}{d\tau}$ requires more work - it describes the evolution of the photon energy, which is described by the geodesic equation. Since we are working to linear order here, the scalar and tensor pertubation contributions decouple, and we can calculate the contribution of each separately. After a slightly tedious calculation we arrive at 

include steps??? got them on a piece of paper somewhere

\begin{equation}
\frac{dp}{d\tau} = p (\Phi' - \hat{p}^i\frac{\partial \Psi}{\partial x^i} -\half h_ij\hat{p}^i\hat{p}^j -\mathcal{H})
\end{equation}

That's all we need. We may decompose any distribution to linear order as $f(x^\mu, p^\mu) = f^{(0)}(p,\tau) + f^{(1)}(\v{x},p,\hat{p}^i,\tau)$. Plugging this, and the above results into \ref{Liouville} we get the following 0'th and 1'st order equations

\begin{equation}\begin{split}
\frac{\partial f^{(0)}}{d\tau} - p\mathcal{H}\frac{\partial f^{(0)}}{dp} &=0\\
\frac{\partial f^{(1)}}{d\tau} + \hat{p}^i\frac{\partial f^{(1)}}{dx^i} - p\mathcal{H}\frac{\partial f^{(1)}}{dp} +p\frac{\partial f^{(0)}}{dp}(\frac{\partial\Phi}{d\tau}-\hat{p}^i\frac{\partial \Psi}{dx^i}-\half \frac{\partial h_{ij}}{d\tau}\hat{p}^i\hat{p}^j) &=0
\end{split}\end{equation}

The zeroth order equation has solution $f^{(0)}(k,\tau) = f^{(0)}(ka)$, explaining the background uniform redshift, as expected. We can fourier transform over the spartial $\v{x}$ dependence of the first order equation, and reintroduce the collision term, obtaining $f = f^{(1)} = f^{(1)}(\v{k}, p, \hat{p}^i, \tau)$

\begin{equation}
f^{(1)'} + ik\mu f^{(1)}-\mathcal{H}p\frac{\partial f^{(1)}}{\partial p} - \frac{\partial f^{(0)}}{\partial k} [ \Phi ' - \Psi ' + ik\mu \Phi + \half h_{ij}'\hat{p}^i\hat{p}^j] = 
C
\end{equation}
where we have introduced $\mu = \unit{k} \cdot \unit{p}$, the angle cosine between photon propagation and fourier mode.

C represents the collision term. For scalar pertubations, it contains source terms proportional to $\unit{p}\cdot\v{v_e}$, for $v_e$ the photon velocity. However, scalar perturbations produce velocities $\v{v} \propto \v{k}$, so we may choose spherical coordinates for $\unit{p}$ to be aligned with $v{k}$. In this case  $f^{(1)}(\v{k}, p, \hat{p}^i, \tau) = f^{(1)}(\v{k}, p, \mu, \tau)$, and is manifestly invariant of $\theta$.

Tensor pertubations do depend on $\phi$. Neglecting any electron velocity arising from tensor pertubations, all the $\phi$ dependence is characterised by the tensor pertubation in the boltzmann equation:

\begin{equation}
h_{ij}'(\v{k},\tau)\hat{p}^i\hat{p}^j = \hat{p}^i\hat{p}^j (h^{+'}(\vec{k},\tau)e_{ij}^+(\vec{k}) + h^{\times'}(\vec{k},\tau)e_{ij}^\times(\vec{k})
\end{equation}

Choosing spherical coordinates with z axis aligned with $\v{k}$, we choose simple polarisation vectors $e^+_{xx}=-e^+_{yy}=1$ and $e^\times_{xy}=e^\times{yx}=1$ with other components 0. From this we calculate using $\hat{k}^i = (\sin\theta\cos\phi, \sin\theta\sin\phi, \cos\phi)$

\begin{equation}\begin{split}
\hat{p}^i\hat{p}^je_{ij}^+ &= \sin^2{\theta}\cos{2\phi} \\
\hat{p}^i\hat{p}^je_{ij}^\times &= \sin^2{\theta}\sin{2\phi}
\end{split}\end{equation}

and so we may write 

\begin{equation}
f^{(1)}(\v{k}, p, \hat{p}^i, \tau) = f^{(1)}(\v{k}, p, \mu, \tau)\cos{2\phi}
\end{equation}

for a + polarized wave


\begin{equation}
f^{(1)}(\v{k}, p, \hat{p}^i, \tau) = f^{(1)}(\v{k}, p, \mu, \tau)\sin{2\phi}
\end{equation}

for a x polarized wave


\section{Old Inflation Chapter}

Inflation is a brief, but very important, period of accelerated expansion in the very early universe, first proposed by [Guth 1981]. It was initially motivated by three problems with the previous standard big bang cosmology, namely the flatness problem (why was the ratio of energy density and critical density so close to unity), the monopole problem, and the horizon problem (why are seemingly casually disconnected regions of the CMB at the same temperature to very high accuracy). Since the birth of the idea, it has become the leading paradigm to the early universe, in part because it provides a quantum mechanical mechanism of generating the primordial density perturbations seeding cosmological evolution. In this chapter we quickly review some important features of inflation.

\subsection{Inflation Basics}

A flat, homogeneous and isotropic universe is described by the Friedmann–Lemaître–Robertson–Walker (FLRW) metric, which in our sign convention takes form

\begin{equation}
\label{FLRW}
ds^2 = - dt^2 + a^2(t)d\v{x}^2 = a^2(\tau)(-d\tau^2+d\v{x}^2)
\end{equation}

and, assuming General Relativity, obeys the Einstein equation $G_{ab} = \frac{1}{\Mp^2} T_{ab}$, sourced by a perfect fluid with energy momentum tensor $T_{ab}$, which by homogeneity and isotropy must take form

\begin{equation}
\label{densityandpressure}
T^0_0 = - \rho(t) \quad T^0_i = 0 \quad T^i_j = P(t)\delta^i_j
\end{equation}

where we  identify $\rho(t)$ as the total energy density and $P(t)$ as the total pressure, ie summed over all fluid components. For our purposes, we will consider one component, the inflaton field, to dominate.  Substituting \ref{FLRW} and \ref{densityandpressure} into the Einstein Equation we obtain the Friedmann equations

\begin{equation}
H^2 = (\frac{\dot{a}}{a})^2 = \frac{1}{3\Mp^2}\rho
\tag{F1}
\label{F1}
\end{equation}
\begin{equation}
\frac{\ddot{a}}{a} = -\frac{1}{6\Mp^2}(\rho + 3P)
\tag{F2}
\label{F2}
\end{equation}

The condition for inflation to occur is \textit{accelerated expansion}, ie $\ddot{a} >0$. Recall the definition of the first hubble slow roll parameter 

\begin{equation}
\label{epsilon}
\epsilon := -\frac{\dot{H}}{H^2} = -\frac{d\ln{H}}{d\ln{a}} = \frac{3}{2}(1+\frac{P}{\rho})
\end{equation}

where the last inequality follows from the Friedmann Equations. We find $\ddot{a} >0$ is equivalent to $\epsilon<1$ and to the condition on the equation of state parameter $\omega=P/\rho < -1/3$.\\ 

In order to solve the horizon problem we require inflation to persist for a relatively long duration of time (~60 e-folds), so $\epsilon$ to must remain small. We parametrise how quickly $\epsilon$ changes in the second hubble slow roll parameter 

\begin{equation}
\eta = -\frac{\dot{\epsilon}}{H\epsilon} = -\frac{d\ln{\epsilon}}{d\ln{a}}
\end{equation}

\subsection{Single Scalar Field Dynamics}

The simplest class of inflation models are those consisting of a single scalar field, slowly rolling down its potential. These postulate the existence of a scalar ``inflation'' field $\phi(t,\v{x})$ with lagrangian density $\mathcal{L} = -\half \partial^\mu \phi \partial_\mu \phi - V(\phi)$ and energy momentum tensor $T_{\mu\nu}= \partial_\mu \phi \partial_\nu \phi + g_{\mu\nu}\mathcal{L}$. \\

Here we consider the classical background evolution, ie take $\phi(t,\v{x}) = \bphi(t)$. There is of course no reason why the field should not also fluctuate spatially, which we consider in the next section. From \ref{densityandpressure} we see that 


SOME MISTAKE HERE OR IN PREVIOUS DEF
\begin{equation}\begin{split}
\rho_\phi &=-T^0_0 = \half \dot{\bphi}^2+V(\bphi)\\
P_\phi&=\frac{1}{3}T^i_i = \half \dot{\bphi}^2-V(\bphi)
\end{split}\end{equation}

Inserting these into the Friedmann equations we get the Klein Gordon Equation

\begin{equation}
\tag{KG}
\label{KG}
\ddot{\bphi}+3H\dot{\bphi}=-V_{,\phi}
\end{equation}

We also find that by \ref{epsilon} that 

\begin{equation}
\epsilon = \frac{1}{\Mp^2}\frac{\half\dot{\bphi^2}}{H^2} < 1
\end{equation}

\subsection{Slow Roll}

The slow roll approximation postulates the kinetic energy and acceleration of the background field is much smaller than its potential energy, which can be encapsulated in terms of our slow roll parameters as $(\epsilon, \eta << 1)$. In this approximation we get by \ref{F1} and \ref{KG} 

\begin{equation}\begin{split}
H^2 &\approx \frac{V}{3\Mp^2} \\
3H\dot{\bphi} &\approx -V_{,\phi}
\end{split}\end{equation}

from which we see 

\begin{equation}
\epsilon \approx \half\Mp^2 (\frac{V'}{V})^2 := \epsilon_V
\end{equation}

where we have defined the first \textit{potential} slow roll parameter. We can anologously define a second potential slow roll parameter via

\begin{equation}
\eta_V = \Mp^2 \frac{V''}{V} \approx 2\epsilon - \half\eta
\end{equation}

From here we may calculate the number of e folds of inflation from some time $t$ until the end of inflation. 

\begin{equation}
N(t) := \ln{\frac{a(t_{end})}{a(t)}} = \int_a^{a(t)} d(\ln{a}) = \int_t^{t_{end}} Hdt = \int_{\bphi(t)}^{\bphi_{end}} \frac{d\bphi}{\sqrt{2\epsilon_V}\Mp}
\label{efolds}
\end{equation}

using $Hdt=\frac{H}{\dot{\bphi}}d\bphi=\frac{d\bphi}{\sqrt{2\epsilon_V}\Mp}$

\subsection{Quantum Fluctuations to $\phi$}

If the inflaton can vary in time, it can also vary in space. The discussion here follows [Baumann]. We consider pertubations over a background

\begin{equation}
\phi(\v{x},\tau) = \bphi(\tau) + \frac{f(\tau, \v{x})}{a(\tau)}
\label{phiexpand}
\end{equation}

We begin with the action for the inflaton, minimally coupled to the metric.

\begin{equation}
S =\int d\tau d^3x \mathcal{L}  =  \int d\tau d^3x \sqrt{-g} (-\half g^{\mu \nu}\partial_\mu \phi\partial_\nu \phi - V(\phi)  
\end{equation}

Plugging in the unperturbed FLRW metric we get 


\begin{equation}
S = \int d\tau d^3x \mathcal{L}  = \int d\tau d^3x \half a^2 [(\phi ' )^2 -(\nabla \phi)^2]-a^4V(\phi)
\label{scalarfieldaction}
\end{equation}

We now plug in \ref{phiexpand} and expand to 2nd order in $f$. The first order piece just gives the Klein Gordon for the background field (in conformal time), as expected. The second order piece gives

\begin{equation}
S^{(2)} = \half \int d\tau d^3x (f')^2 - (\nabla f)^2 + (\frac{a''}{a}-a^2V'')f^2
\end{equation}

after integrating by parts and making use of \ref{F2} in conformal time. In the slow roll approximation we may drop the potential term since it is slow roll suppressed compared to the other terms.

and so 

\begin{equation}
S^{(2)} \approx \half \int d\tau d^3x (f')^2 - (\nabla f)^2 + \frac{a''}{a}f^2
\end{equation}

Note since we have dropped the potential entirely, this is just the second order action for a massless scalar field. Integrating by parts and demanding $S^{(2)}=0$ gives the Muhkanov Sasaki equation, which we can write in real or fourier space:

\begin{equation}\begin{split}
f''-\nabla^2f-\frac{a''}{a}f &= 0 \\
\Leftrightarrow f''_{\v{k}} + (k^2-\frac{a''}{a})f_{\v{k}} &= 0
\label{MS}
\end{split}\end{equation}

We treat these perturbations $f$ quantum mechanically, and so require the techniques of QFT on curved spacetimes. We'll outline the key steps in quantising this system.\\

The conjugate momentum to $f$ is $\pi(\tau, \v{x}) =  \frac{\partial \mathcal{L}}{\partial f'} = f'$ using \ref{scalarfieldaction}. We promote these to operators $\hat{f}(\tau, \v{x})$ and $\hat{\pi}(\tau, \v{x})$ satisfying equal time commutation relations which read in real and fourier space:

\begin{equation}\begin{split}
[\hat{f}(\tau, \v{x}), \hat{\pi}(\tau, \v{x'}] &= i\delta(\v{x}-\v{x'}) \\
[\hat{f}_{\v{k}}(\tau), \hat{\pi}_{\v{k'}}(\tau)] &= (2\pi)^3i\delta(\v{k}+\v{k'})
\end{split}\end{equation}

We mode expand $\hat{f}_{\v{k}}(\tau) = f_k(\tau)\ann{k}+f_k^*(\tau)\cre{k}$, demanding the modefunctions $f_k(\tau)$ and $f_k^*(\tau)$ are two linearly independent solutions of the Muhkanov-Sasaki equation. Substituting into the commutation relations we get 

\begin{equation}
W[f_k,f_k^*]\times[\ann{k}, \cre{k}] = (2\pi)^3\delta(\v{k}+\v{k'})
\end{equation}

which after normalising the Wronskian to 1 gives the usual commutator of annhilation and creation operators

\begin{equation}
[\ann{k}, \cre{k}] = (2\pi)^3\delta(\v{k}+\v{k'})
\end{equation}

We can now define the Hilbert space as the usual Fock space formed by unions of n particle states obtained by applying n creation operators to the the vacuum, satisfying 

\begin{equation}
\ann{k}\vac =0 \qquad \forall \v{k}
\end{equation}

Note this doesn't completely fix the vacuum, since we have not yet fixed our mode functions. We construct the Bunch-Davies vacuum, by imposing the mode functions must be positive frequency \footnote{this is required for our hilbert space to only consist of positive norm states, and is a common requirement in many quantum field theories}, and also match the minkowski mode functions $f_k(\tau) \propto e^{\pm ik\tau}$ at early times, since at early times $\tau \rightarrow -\infty$

\begin{equation}
\ref{MS} \rightarrow  f''_{\v{k}} + (k^2)f_{\v{k}} = 0
\end{equation}

We further make the quasi-deSitter approximation, where H is constant and $a=-\frac{1}{H\tau}$, and so \ref{MS} becomes

\begin{equation}
f''_k + (k^2-\frac{2}{\tau^2})f_k = 0
\end{equation}

with general solution

\begin{equation}
f_k(\tau) = \alpha \frac{e^{-ik\tau}}{\sqrt{2k}}{(1-\frac{i}{k\tau}} + \beta \frac{e^{ik\tau}}{\sqrt{2k}}{(1+\frac{i}{k\tau}})
\end{equation}

which matches our initial condition for $\beta = 0$ and $\alpha=1$, giving the Bunch Davies mode function 

\begin{equation}
f_k(\tau) = \frac{e^{-ik\tau}}{\sqrt{2k}}{(1-\frac{i}{k\tau}})
\end{equation}

The key result of this section is the power spectrum of fluctuations, given by 

%We now have all we require to calcuate the spatial variance, or two point correlator $\varepsilon(0)$ of $f$ and therefore of $\delta \phi = f/a$. First recall the relevant definitions:
%
%the two point correlation function of a field f is given by at fixed (omitted) time $\tau$ by
%
%\begin{equation} 
%\varepsilon_f(\v{x})f(\v{y}) = <f(\v{x})f(\v{y})> = <0|f(\v{x}f(\v{y})|0>
%\end{equation}
%
%Assuming statistical homogeneity and isotropy, this depends only on $|\v{x}-\v{y}|$, and is related to the power spectrum:

\begin{equation}
\langle f_{\v{k}}f_{\v{k'}} \rangle = \langle 0|f_{\v{k}}f_{\v{k'}}|0\rangle =(2\pi)^3\delta(\v{k}+\v{k'})P_f(k)\
\end{equation}

%by
%
%\begin{equation}
%\varepsilon(r)=\varepsilon(\v{x},\v{x}+\v{r})=\fint{k} P_f(k)e^{i\v{k}\cdot\v{r}}
%\end{equation}


The power spectrum of our field is easily computed to be 

\begin{equation}
\langle f_{\v{k}}f_{\v{k'}}\rangle = \langle (f_k(\tau)\ann{k}+f_k^*(\tau)\cre{k})(f_k(\tau)\ann{k'}+f_k^*(\tau)\cre{k'}\rangle = |f_k|^2
\end{equation}

and so the dimensionless power spectra of interest are  

\begin{equation}\begin{split}
\Delta^2_f(k):=\frac{k^3}{2\pi^2}P_f(k) &= \frac{k^3}{2\pi^2}|f_k|^2\\
\Rightarrow \Delta^2_{\delta\phi}(k) &=\frac{k^3}{2\pi^2a^2}|f_k|^2\\
&=\frac{k^2}{4\pi^2a^2}(1+\frac{1}{k^2\tau^2})\\
&=(\frac{H}{2\pi})^2(1+\frac{k^2}{a^2H^2} \text{using $a=\frac{-1}{H\tau}$}\\
&\rightarrow \frac{H}{2\pi})^2 \text{on superhorizon scales $k<<\mathcal{H}$}
\label{inflatonpower}
\end{split}\end{equation}

We approximate the power spectrum at horizon crossing to be 

\begin{equation}
\Delta^2_{\delta\phi}(k) \approx (\frac{H}{2\pi})^2\rvert_{k=aH}
\end{equation}

We will return to this result later.








\section{Old boltzmann eqn bit}

OLD


\begin{equation}
\Delta_s(\v{k}, \v{p}, \tau) = 4f^{(1)}_s(\frac{\partial\bar{f}_I}{\partial \ln{T}})^{-1}
\end{equation}





\section{Old tensor power spectrum}

Since the form of these are so similar, we can compute the TT, EE and BB correlation functions all in one go. We can also, with more work, verify that T does not correlate with E or B.\\

We'll begin by calculating the TT power spectrum $C_l^{TT}$ sourced by tensor pertubations. Recalling the definition of the angular power spectrum, we first require the spherical multipole coefficients $T_{lm}$  

\begin{equation}
T_{lm} = \int d\Omega Y_{lm}^*(\unit{n})\Delta_T(\tau_0,\unit{n})
\end{equation}

From \ref{powerspectra} we can write 

\begin{equation}\begin{split}
C_l^{TT} &= \frac{1}{2l+1}\sum_m \langle T_{lm}^*T_{lm} \rangle \\
&= \frac{1}{2l+1} \sum_m [\int d\Omega Y_{lm}^*(\unit{n})\int d^3k\int_0^{\tau_0}d\tau S_T(\tau,k)e^{ix\mu}]^*
[\int d\Omega Y_{lm}^*(\tilde{\unit{n}})\int d^3\tilde{k}\int_0^{\tau_0}d\tilde{\tau}S_T(\tau,\tilde{k})e^{i\tilde{x}\tilde{\mu}}]\\
&\times
\langle [(1-\mu^2) e^{2i\phi} \xi^1(\v{k})+(1-\mu^2) e^{-2i\phi} \xi^2(\v{k})]^*[(1-\tilde{\mu}^2) e^{2i\tilde{\phi}} \xi^1(\v{\tilde{k}})+(1-\tilde{\mu}^2) e^{-2i\tilde{\phi}} \xi^2(\v{\tilde{k}})] \rangle
\end{split}\end{equation}

The expectation value evaluates to

\begin{equation}\begin{split}
&(1-\mu^2)(1-\tilde{\mu}^2)[\langle\xi^{1*}(\v{k})\xi^1(\v{\tilde{k}})\rangle e^{2i\tilde{\phi}-2i\phi}+\langle\xi^{2*}(\v{k})\xi^2(\v{\tilde{k}})\rangle e^{-2i\tilde{\phi}+2i\phi}]\\
&~ (1-\mu^2)(1-\tilde{\mu}^2)P_t(k)\delta(\v{k}-\v{\tilde{k}})\half(e^{2i\tilde{\phi}-2i\phi}+e^{2i\phi-2i\tilde{\phi}})
\end{split}\end{equation}

and so the final expression can be written conveniently as 

\begin{equation}\begin{split}
C_l^{TT} = \frac{1}{2l+1} \int k^2 dk P_t(k) \sum_m \bigg| \int d\Omega Y^*_{lm}(\unit{n}) \int_0^{\tau_0} d\tau  S_T(k,\tau)(1-\mu^2)e^{2i\phi}e^{ix\mu} \bigg|^2
\end{split}\end{equation}

To evaluate the angular integrals, we use the expression $Y_{lm} = [\frac{(2l+1)(l-m)!}{4\pi(l+m)!}]^\half P_l^m(\mu)e^{-im\phi}$ for spherical harmonics in terms of legendre polynomials, and note that the $\phi$ integral vanishes unless $m=2$, in which case it gives $2\pi$. So we get 

\begin{equation}\begin{split}
C_l^{TT} = 4\pi^2\ltwof \int k^2 dk P_t(k) \bigg|  \int_0^{\tau_0} d\tau  S_T(k,\tau)\int_{-1}^1 d\mu P_l^2(\mu)(1-\mu^2)e^{ix\mu} \bigg|^2
\end{split}\end{equation}

Isolating the $\mu$ integral

\begin{equation}\begin{split}
\int d\mu P_l^2(\mu)(1-\mu^2)e^{ix\mu} 
&= \int d\mu(1-\mu^2)^2 \partial_\mu^2P_l(\mu)e^{ix\mu}\\
&= \int d\mu \partial_\mu^2P_l(\mu)(1+\partial_x^2)^2e^{ix\mu}\\
&= \int d\mu P_l(\mu)(1+\partial_x^2)^2(x^2e^{ix\mu})\\
&= 2i^l(1+\partial_x^2)^2(x^2j_l(x))\\
&= 2i^l(\frac{j_l(x)}{x^2})
\end{split}\end{equation}

making use $P^m_l(\mu)=(-1)^m(1-\mu^2)^{m/2}(\partial_\mu)^mP_l(\mu)$, $\int d\Omega Y_{lm}^*(\unit{n})e^{ix\mu} = \sqrt{4\pi(2l+1)}i^l j_l(x) \delta_{m0}$ and $Y_{l0} = \sqrt{\frac{2l+1}{4\pi}}P_l(\mu)$. In the final line we used the ODE for spherical bessel functions: $j_l''+\frac{2j_l'}{x}+(1-\frac{l(l+1)}{x^2}j_l)=0$.\\

Thus our final result is:

\begin{equation}\begin{split}
C_l^{TT} = (4\pi)^2\ltwof \int k^2 dk P_t(k) \bigg|  \int_0^{\tau_0}d\tau  S_T(k,\tau)\frac{j_l(x)}{x^2} \bigg|^2
\end{split}\end{equation}

Now we abuse the similarity of expressions in \ref{LoSfourier} to write down the EE and BB power spectra. The angular dependence of $\Delta_{\tilde{E}}$ and $\Delta_{\tilde{B}}$ are exactly the same as those of $\Delta_T$. The expressions differ only in the $\mathcal{E}$ and $\mathcal{B}$ operators, acting seperately on each $\v{k}$ mode, which can be applied after the angular integrals are performed. We lose the $l$ dependent factor in front by converting from $\tilde{E}, \tilde{B}$ to $E,B$, using \ref{EBtwiddle}


\begin{equation}\begin{split}
C_l^{EE} &= (4\pi)^2 \int k^2 dk P_t(k) \bigg|  \int_0^{\tau_0}d\tau  S_P(k,\tau)\mathcal{E}(x)\frac{j_l(x)}{x^2} \bigg|^2\\
&= (4\pi)^2\int k^2 dk P_t(k) \bigg|  \int_0^{\tau_0}d\tau  g(\tau)\Psi(k, \tau)[-j_l(x) +j_l''(x)+\frac{2j_l(x)}{x^2} + \frac{4j_l'(x)}{x}]\bigg|^2\\
C_l^{BB} &= (4\pi)^2 \int k^2 dk P_t(k) \bigg|  \int_0^{\tau_0}d\tau  S_P(k,\tau)\mathcal{B}(x)\frac{j_l(x)}{x^2} \bigg|^2\\
&= (4\pi)^2\int k^2 dk P_t(k) \bigg|  \int_0^{\tau_0} d\tau g(\tau)\Psi(k, \tau)[2j_l'(x)+\frac{4j_l(x)}{x}]\bigg|^2
\label{primordialBmodes}
\end{split}\end{equation}












\section{Old Boltzmann equation derivation that i dont like}

To completely understand the full photon distribution we must follow the evolution of four distribution functions, expressed as a column vector $f$. These may be taken to be the four Stokes parameters, but following [Zhao and Zeng] we instead use $f=(I_l, I_r, U, V)$ with $I=I_l+I_r$ and $Q=I_l-I_r$. Prior to decoupling, Thompson scattering of anistropic radiation by free electrons gives rise to only linear polarization, so we drop $V$, so that $f=(I_l, I_r, U)$. For homogeneous and isotropic unpolarised radiation the distribution is simply $f=f_0(\nu)(1,1,0)$, where $f_0(\nu)=[e^{\hbar\nu/kT}-1]^{-1}$, the blackbody Planck distribution. Metric perturbations and Thompson scattering break this, and yield linear polarization. The time evolution of the the photon distribution $f$ is given by the radiative transfer equation, which is essentially the Boltzmann equation 

\begin{equation}
\frac{\partial f}{\partial \tau} + \unit{n}^i\frac{\partial f}{\partial x^i} = - \frac{d\nu}{d\tau}\frac{\partial f}{\partial \nu} - q(f-J)
\end{equation}

where $\unit{n}$ is the unit vector in the direction of photon propagation $(\theta, \phi)$, $q=\sigma_tn_ea$ is the differential optical depth and

\begin{equation}
J=\frac{1}{4\pi} \int_{-1}^1 d\mu'\int_0^{2\pi} d\phi' P(\mu,\phi,\mu'\,\phi')f(\tau, x^i, \nu, \mu', \phi')
\end{equation}

for $\mu=\cos\theta$ and $\mu'=\cos\theta'$ and P is the scattering matrix (add later). This equation described quite basic physics: It says the total time derivative of the photon distribution (written as an explicit time derive plus the time evolution due to photon motion) is given by gravitational red-shift in the perturbed spacetime plus a change due to scattering of electrons. The quantity J is the angular intensity polarization distribution that arises after the radiation has been Thomson scattered once from an initial distribution f. The scattering matrix looks messy, and is derived from the angular polarization dependence of thomson scattering.

\begin{equation}
\frac{d\sigma_T}{d\Omega} \propto |\hat\varepsilon\cdot\hat\varepsilon'|^2
\end{equation}

The term $- \frac{d\nu}{d\tau}\frac{\partial f}{\partial \nu}$ reflects the variation of frequency due to metric perturbations. This can be calculated from the geodesic equation and is given by

\begin{equation}
\frac{1}{\nu}\frac{d\nu}{d\tau} = \half \frac{\partial h_{ij}}{d\tau}\unit{n}^i\unit{n}^j
\end{equation}

We work in conformal time, in which the perturbed metric can be written as 

\begin{equation}
ds^2 = a^2(\tau)(d\tau^2-(\delta_{ij}+h_{ij})dx^idx^j)
\end{equation}

For our purposes we only consider tensor perturbations. Choosing spherical coordinates with z axis aligned with the propagation axis $\unit{n}$, we choose simple polarisation vectors $e^+_{xx}=-e^+_{yy}=1$ and $e^\times_{xy}=e^\times{yx}=1$ with other components 0. From this we calculate using $\hat{n}^i = (\sin\theta\cos\phi, \sin\theta\sin\phi, \cos\phi)$

\begin{equation}\begin{split}
\hat{p}^i\hat{p}^je_{ij}^+ &= \sin^2{\theta}\cos{2\phi} \\
\hat{p}^i\hat{p}^je_{ij}^\times &= \sin^2{\theta}\sin{2\phi}
\end{split}\end{equation}

Now in the presence of such perturbations the distribution function will be perturbed, and may be written

\begin{equation}
f(\theta,\phi) = f_0 \begin{pmatrix}
1\\
1\\
0
\end{pmatrix} + f_1
\end{equation}

In cosmology, we take the amplitude of the two components to be the same, and them to have the same statistical properties. Following an approach taken first by Polnarev, we introduce variables suited to this angular dependence, and also decomposing the distribution perturbation into an anisotropy and polarisation part. Firstly for the + polarisation:

\begin{equation}
f_1 = \frac{\Tilde{\Delta}_T}{2}(1-\mu^2)\cos{2\phi}\begin{pmatrix}1\\1\\0\end{pmatrix} + \frac{\Tilde{\Delta}_P}{2}\begin{pmatrix}(1+\mu^2)\cos{2\phi}\\-(1+\mu^2)\cos{2\phi}\\4\mu\sin{2\phi}\end{pmatrix}
\end{equation}

and for the x polarisation

\begin{equation}
f_1 = \frac{\Tilde{\Delta}_T}{2}(1-\mu^2)\sin{2\phi}\begin{pmatrix}1\\1\\0\end{pmatrix} + \frac{\Tilde{\Delta}_P}{2}\begin{pmatrix}(1+\mu^2)\sin{2\phi}\\-(1+\mu^2)\sin{2\phi}\\4\mu\cos{2\phi}\end{pmatrix}
\end{equation}

We see $\Tilde{\Delta}_T \propto I_l+I-r = I$ corresponds to the isotropy and $\Tilde{\Delta}_P \propto I_l - I_r = Q$ and $\Tilde{\Delta}_P \propto U$. They are determined by solving the Boltzmann equation. We'll work in Fourier space throughout, focussing on a mode with wave vector $\v{k}$, which we may choose to be aligned with the $\unit{z}$ direction, and take the basis on which we evaluate $Q$ and $U$ to be the natural basis on the sphere: $(\unit{e_1}, \unit{e_2}) = (\unit{e_\theta}, \unit{e_\phi})$. These take the same form for both + and x, and are obtained by substituting the full form of $f$ into REF, taking the fourier transform, and performing the angular integrals. Retaining only linear order terms we obtain 2 equations and omitting the $k$ dependence we arive at 

\begin{equation}\begin{split}
\tilde{\Delta}_T'+ik\mu \tilde{\Delta}_T &= -h' -\kappa'[\tilde{\Delta}_T - \Psi]\\
\tilde{\Delta}_P'+ik\mu\tilde{\Delta}_P &= -\kappa'[\tilde{\Delta}_P + \Psi]
\end{split}
\label{Boltzmann}\end{equation}

where the source is

\begin{equation}
\Psi = \frac{1}{10}\tilde{\Delta}_{T0} + \frac{1}{7}\tilde{\Delta}_{T2} + \frac{3}{70}\tilde{\Delta}_{T4} - \frac{3}{5}\tilde{\Delta}_{P0} + \frac{6}{7}\tilde{\Delta}_{P2} - \frac{3}{70}\tilde{\Delta}_{P4}
\end{equation}


written in terms of multipole moments, defined for any function as $f(k, \mu) = \sum_l (2l+1)(-i)^lf_l(k)P_l(\mu)$ for $P_l(\mu)$ the order $l$ legendre polynomial. All derivatives are taken with respect to conformal time $\tau$. $h$ is the gravitational wave amplitude.\\

The \textit{optical depth} for Thomson scattering is given by the integral $\kappa(\tau) = \int_\tau^{\tau_0} an_e\sigma_T$ and denotes the probability for a photon not to scatter between conformal time $\tau$ and today $\tau_0$ as $e^{-\kappa}$. From this we see $\kappa'=-an_e\sigma_T$. Let us also introduce the \textit{visibility function} $g(\tau) = \kappa'e^{-\kappa}$, encoding the probability a photon last scattered at $\tau$. Integrating:

\begin{equation}
\int_0^{\tau_0} g d\tilde{\tau} = [-e^{-\kappa}]_0^{\tau_0} = 1
\end{equation}

using the limits $\kappa \rightarrow \infty$ as $\tau\rightarrow0$ (early times) and $\kappa \rightarrow 0$ as $\tau\rightarrow \infty$ (late times).\\




 The \textit{differential optical depth} for Thomson scattering is $-\kappa'=$, where $n_e$ is the electron density, and $\sigma_T$ is the Thomson cross section.  \\
something fishy with minus signs in optical depths here\\


From our above expressions we may calculate the perturbations of $T, Q, U$ as, for a + polarised wave:

\begin{equation}
\begin{split}
\Delta_Q  &= \Tilde{\Delta}_P(1+\mu^2)\cos2\phi\\
\Delta_U  &= \Tilde{\Delta}_P 2\mu\sin2\phi\\
\Delta_T  &= \Tilde{\Delta}_T(1-\mu^2)\cos2\phi
\end{split}
\end{equation}

whereas for a x polarised wave

\begin{equation}
\begin{split}
\Delta_Q  &= \Tilde{\Delta}_P(1+\mu^2)\sin2\phi\\
\Delta_U  &= -\Tilde{\Delta}_P2\mu\cos2\phi\\
\Delta_T  &= \Tilde{\Delta}_T(1-\mu^2)\sin2\phi
\end{split}
\end{equation}